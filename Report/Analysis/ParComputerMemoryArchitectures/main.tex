\section{Memory Architectures}

This section describes two memory architectures commonly used in parallel computing. Based on the characteristics of the architectures, a target architecture is chosen for TLDR.

\subsection{Shared Memory}

In this memory architecture, all processors in the computer have access to the same memory, that is, the memory address space is global. In addition the memory is in close proximity to the processors, providing lower access latency. This approach of having unified memory is very simplistic and eases the programmer's view of the memory layout.
Because of this sharing of memory, a processor A can affect memory used by processor B. This leads to non-deterministic programs where a processor can not be sure what state the memory is in at a given time, without synchronisation between processors~\cite{compLLNL}. 

To summarise on the above, the architecture has the following advantages and disadvantages. 

\noindent\textbf{Advantages}
\begin{itemize}
    \item Simple unified memory model leads to simple understanding of the memory layout
    \item Fast access to memory
\end{itemize}

\noindent\textbf{Disadvantages}
\begin{itemize}
    \item Sharing of memory between processes leads to non-deterministic processes if synchronisation between processes is not employed
\end{itemize}

\subsection{Distributed Memory}

In this memory architecture, processors have their own local memory, and it is not possible to directly access memory between processors.
For two processors to share memory, the data must be sent via a connection. A typical connection method is Ethernet. A system of processors is called a distributed system once connected through a network. Depending on the proximity of processors and the bandwidth of the connection method, the speed and latency may be sub-par compared to the shared memory architecture, described above~\cite{compLLNL}.

To summarise, the distributed architecture has the following advantages and disadvantages.

\noindent\textbf{Advantages}
\begin{itemize}
    \item It is easy to add new processors to the network, since each processor only known about its own memory. The network therefore scales well
\end{itemize}

\noindent\textbf{Disadvantages}
\begin{itemize}
    \item It is more complex to access data on another processor, and data located on a remote processor is slower to retrieve
\end{itemize}

\subsection{Architecture Choice}

TLDR is targeted at scientists wanting to perform simulations. Because of the need to perform these simulations quickly, even without super-computers, there should be support for connecting several computers in a network to create a distributed system. The distributed memory architecture is therefore chosen as the preferred architecture choice for this programming language.
