\url{http://gribblelab.org/CBootcamp/A2_Parallel_Programming_in_C.html#sec-2-1}
\url{https://computing.llnl.gov/tutorials/parallel_comp/}

\section{Types of tasks}
\label{top}

Tasks within a problem can be depended on each other, in the sense that one task needs the output of a computation done by another task. This section will describe two types of problems relevant when doing parallel computations.

\subsection{Embarrasingly parallel problems}
  A problem can be describe as being \emph{embarrasingly parallel} when the tasks within the problem is independent of each other, and can therefore be parallelised without any concern of the order in which the task can to be executed. They are trivially parallel in nature because of the independency.
  An example of this type of problem is incrementation of a large matrix, the individuals members of matrix are totally independent from each other under operations of these tasks.

\subsection{Serial problems with dependencies}
  Although multiple similar simulations can be observed as being independent of each other, as utilised by the Monte Carlo method \url{http://en.wikipedia.org/wiki/Monte_Carlo_method}, most simulations themselfs does not satify the condition of being independent. Instead these are inherently sequential, they form a class of problems that cannot be split into independent sub-problems. In some cases it is not possible to gain speed up at all, by trying to parallelise a problem, which is not parallisesable, the only thing a simulation designer can achieve is adding overhead to the computations in communication between threads.
  An example is calculating the fibonacci series by f(n) = f(n-1)+f(n-2), where f(n) is dependent on first finding the previously calculated values of f.
