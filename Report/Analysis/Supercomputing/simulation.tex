\section{Models}
Model is a term often used in both natural and social sciences. Unfortunately the term is not as well defined as it perhaps should be, and it will therefore here be specified what is meant by it. The approach here is based on Mario Bunge definition of a model \kanote{indsæt kilde}. According to his theory a model consists of two parts:
\begin{itemize}
\item A general theory
\item A special description of an object or system (model object)
\end{itemize}
This perception is a very general way of thinking of models and often works very well with natural and, to some degree, social sciences where experiments and models are based on an overall theory and consists of objects and systems. In social sciences, though, it often happens that the general theory is vaguely defined or even non-existant and the purpose of the model is only to observe interactions and behaviour of the systems and their objects.
To extend Bunges theory models will here be defined as \enquote{a set of assumptions about some system} meaning everything we know about a system that we can observe or a theoretical system.
Genrally there are two different types of models, static and dynamic models.

\subsection{Dynamic Models}
Dynamic models includes some form of evolution, meaning that the system changes due to some changing factor. This is often time, but can also be things like energy, as is often the case in chemistry, or alike, the only important thing is that the factor changes. Nearly all systems in natural and social sciences are dynamic models.

\subsection{Static Models}
A static model is a snapshot of a given system with objects with non-changing states. These systems are often not very representative for reality since nearly all systems changes but they can still be very usefull to construct since it can be much easier to collect information and understanding through a static model. 

\section{Simulation}
We often need to create and analyse models. For a portion of these models an analytical approach can be used, where the one can analyse the model via mathematical methods and calculate how static models will look and dynamic models evolve. This approach only works on a limited portion of the models where the system and behaviour of objects are well known and can be described mathematically. As the models become more stochastic, purely mathematical analysis begins to fall short and simulations are often used to be able to create the (often dynamic) models.

Formally simulations is defined as \enquote{a system designed to imitate another real system} \jenote{Find a better definition of a simulation}. This is a very broad definition, since real systems can be very different in their structure and further more the type of simulations also differs greatly. Generally one can distinguish between two different types of simulations, experimental and theoretical. An experimental simulation is where a real system is imitated by another real system, often smaller and/or simpler. For instance, when biologists simulate the creation of life on the back of crystals in their lab it is an experimental simulation. Theoretical simulations are simulations where a real system is imitated theoretically often on a computer or on paper. This project will focus on theoretical simulations. To further narrow the definition of a simulation used in this project, the focus will mainly be put on natural and social sciences where simulations are most often used. Within these sciences real systems can still vary a lot, but they nearly always share the property that they change state over, what is often but not necessarily, time. This is close to the definition of a dynamic model. In this respect simulations are close to a dynamic model. How this change is modelled can also vary and generally we often speak of two different paradigms; Discrete and continuous simulations.

\subsection{Discrete Simulations}
Discrete simulations are often dictated by flows and states where the state of the whole system and subsystems are updated in discrete iterations. The systems and objects update their state each iteration. Observations in discrete simulations does not make sense during update but only before/after iterations. Figure \jenote{Make figure} shows an illustration of discrete simulations. Here we can see that the model updates instantaneous between iterations. This abstractions tend to be more useful when the simulations are better known and knowledge exists on how iterations are set in motion and the state is updated.

\subsection{Continuous Simulations}
Continuous simulations are simulations that can be described with differential change. This means that 

% We make for ourselves internal images or symbols of the external objects, and we make them in such a way that the consequences of the images that are necessary in thought are always images of the consequences of the depicted objects that are necessary in nature ... Once we have succeeded in deriving from accumulated previous experience images with the required property, we can quickly develop from them, as if from models, the consequences that in the external world will occur only over an extended period or as a result of our own intervention.
\label{simulationchoise}
As an abstraction for the language to create simulations

\subsection{The Iteration Problem}
By choosing the actor model it is, as explained in \cref{simulationchoise}, possible to create continuous simulations in the language with a good abstraction for simulations and models in general. Unfortunately the actor model does not support discrete simulations very well, and when trying to create discrete simulations with this, a problem arises that needs to be addressed.

The problem is that there is no way for the programmer in the language to tell whether a group of actors has stopped working or not. This is important since the user must to be able to implement iteration steps if he or she is to create discrete simulations. He or she has to be able to find out when one step has ended to update the state of the whole system and start a new iteration. Due to the parallel nature of the actor model an unfortunate variant of race conditions arise.
For an actor to have stopped the programmer needs to know the three following:
\begin{itemize}
\item The actor is not currently executing any messages
\item The actors message queue is empty
\item The actor will not receive any further messages
\end{itemize}
The two first conditions are inherently easy to check. But for the last condition to hold one must make sure that all other actors that know about this actor also have stopped for else they can send messages to the actor and thereby \enquote{reviving} it. First of all the programmer must be able to group actors together thereby making sure that only actors in these groups can communicate with each other. This is done so he or she can know what actors must have stopped.

In this project the scope of parallelisation is within simulations. The rest of this section will describe some general properties and processes of computer simulations. At last this section will discuss generalisations and assumptions, that can be made on the basis of the understading of the problem domain. This can be used to provide the design phase with some ideas of abstractions the language should implement.

\emph{Understanding the problem domain}

Firstly before spending time in an attempt to develop a parallel solution for any problem, one should first determine whether or not the problem is one that can actually be parallelized. This will be described in \cref{top}.

\emph{Granuality}

In order to do a computable simulation of a problem, a level of granuality is to be determined, this is a significant for both computation time and accuracy of the simuilation. A description of this process is giving in \cref{dis}

\emph{Helping processes and tools for making solutions parallel}

Designing and developing parallel programs is characteristically a very manual process. The programmer is typically responsible for both identifying and actually implementing parallelism.

Manually developing parallel solutions is a time consuming, complex, error-prone and an iterative process.

Various tools can assist the programmer with converting serial programs into parallel programs. The most common type of tool used to automatically parallelize a serial program is a parallelizing compiler or pre-processor.

This is usually a process of the compiler analysing the source code of a serial solution and identifying opportunities for parallelism. The analysis includes identifying inhibitors to parallelism and possibly a cost weighting on whether or not the parallelism would actually improve performance. For example loops (do, for) are the most frequent subject to automatic parallelization.
