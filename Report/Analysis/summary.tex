%mainfile: ../master.tex
\section{Criteria for Language Design}\label{analsum}

This section will, in light of the preceding two chapters, \cref{part:analysis} and \cref{part:analysis2}, settle on some criteria for TLDR. The criteria are inspired by those found in \cite{sebesta2015concepts}. However, only those characteristics which differentiate TLDR will be discussed, even though other characteristics will be present in the language.


\subsection{Simplicity, Orthogonality and Syntax Design}

In order to accommodate a user group with programming as secondary skill sets, TLDR should be simple. This would result in strict and straightforward rules regarding interactions. The language should try to keep simple rules regarding orthogonality as well, but not necessarily be highly orthogonal for that reason. One such rule, based on properties of the actor model, is that actors should, in the eyes of the programmer, only communicate through messages either sent or received. This would prohibit directly addressing fields in an actor, which would lower orthogonality, but keep the language simple. The syntax design, should for the same reasons, cater towards a modelling perspective and not necessarily a technical accurate perspective.

\subsection{Data Types}

Since the users of TLDR are will typically have a background in mathematics, or require to include some mathematics in order to asses the value of any given result of a simulation, the language should strive towards having data types which allow for representation of traditional mathematical numbers.


\subsection{Type Checking and Exception Handling}

Due to the language targeting high performance computation, the language should try to avoid run-time errors, and catch problems as early as possible. This suggests a strongly typed language, but we will elaborate further on this matter in \cref{typesys}. For the same reasons, the language will not value exception handling, since edge cases should be fully encompassed in a simulation. However it is worth noticing that the actor model could require exception handling, if communication problems, caused by either race conditions or network conditions, should prove frequent.


%\subsection{Support for abstraction}


%\subsection{Aliasing}
