\subsection{Scoping}

As with most of the choices made during the design of this language, we wanted to strive towards a natural mathematical syntax. This idea was pulling towards the use of linebreaks for denoting the end of a statement, and using indentation for denoting scopes. However since this is impossible to describe with context-free-grammar, and explicitly modifying the parser to support this was adding unnecessarily complexity to the language implementation, we decided upon other constructs.
%impossible - but why?
For denoting scopes we went with the bracket symbols \enquote{\{ \}}. This decision was made due to its similarity with the parentheses know from mathematics, where it has the highest precedence, and also since it is a construct known from many other programming languages. For separating statements, we went with a semi-colon \enquote{;}. This is also a known construct from other programming languages. It is also not in conflict with mathematical use, since there is no widespread consensus of such.

Here are the semantics for the block statement:

\begin{align*}
&\inference[$\text{BLOCK}_1$]{\Braket{S,e} \Rightarrow_S \Braket{S',e'}}
                         {\Braket{\{S\},e} \Rightarrow_S \Braket{\{S'\},e'}}
\\\\
&\inference[$\text{BLOCK}_1$]{\Braket{S,e} \Rightarrow_S e'}
                         {\Braket{\{S\},e} \Rightarrow_S e'}
\end{align*}