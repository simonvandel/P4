\section{Actors}

Here follows descriptions of the usage of actors in TLDR. This includes different principles, functionalities, syntactical and the semantics. But before we can discuss the use of actors in TLDR, we must first cover the semantics of the parallelism used.

\subsubsection{Semantics of parallelism}
Parallelism is achieved through non-determinism in TLDR. The following four rules all have the same left side of the arrow in the conclusion meaning that for the statement

\begin{align*}
%%%%%%%%%%%%%%%%%%%%%%%%%%%%%%%%%%%%%%%%%%%%%%%%%%%%%%%%%%%%%%%
\intertext{The left side of a parallel statement is executed, but not finished. The environment and actor model is updated when executing $S_1$.}
&\inference[$\text{PAR}_1$]{\Braket{S_1,e_1,\Ta} \Rightarrow_S \Braket{S_1',e_1',\Ta'}} 
                           {\Braket{S_1,e_1,\Ta}|\Braket{S_2,e_2,\Ta} \Rightarrow_S \Braket{S_1',e_1',\Ta'}|\Braket{S_2,e_2,\Ta'}}
%%%%%%%%%%%%%%%%%%%%%%%%%%%%%%%%%%%%%%%%%%%%%%%%%%%%%%%%%%%%%%%
\intertext{The left side of a parallel statement is executed, and finishes. The environment and actor model is updated when executing $S_1$.}
&\inference[$\text{PAR}_2$]{\Braket{S_1,e_1,\Ta} \Rightarrow_S \Braket{e_1',\Ta'}} 
                           {\Braket{S_1,e_1,\Ta}|\Braket{S_2,e_2,\Ta} \Rightarrow_S \Braket{S_2,e_2,\Ta'}}
%%%%%%%%%%%%%%%%%%%%%%%%%%%%%%%%%%%%%%%%%%%%%%%%%%%%%%%%%%%%%%%
\intertext{The right side of a parallel statement is executed, but not finished. The environment and actor model is updated when executing $S_2$.}
&\inference[$\text{PAR}_3$]{\Braket{S_2,e_2,\Ta} \Rightarrow_S \Braket{S_2',e_2',\Ta'}} 
                           {\Braket{S_1,e_1,\Ta}|\Braket{S_2,e_2,\Ta} \Rightarrow_S \Braket{S_1,e_1,\Ta'}|\Braket{S_2',e_2',\Ta'}}
%%%%%%%%%%%%%%%%%%%%%%%%%%%%%%%%%%%%%%%%%%%%%%%%%%%%%%%%%%%%%%%
\intertext{The right side of a parallel statement is executed, and finishes. The environment and actor model is updated when executing $S_2$.}
&\inference[$\text{PAR}_4$]{\Braket{S_2,e_2,\Ta} \Rightarrow_S \Braket{e_2',\Ta'}}
                           {\Braket{S_1,e_1,\Ta}|\Braket{S_2,e_2,\Ta} \Rightarrow_S \Braket{S_1,e_1,\Ta'}}
%%%%%%%%%%%%%%%%%%%%%%%%%%%%%%%%%%%%%%%%%%%%%%%%%%%%%%%%%%%%%%%
\end{align*}

\subsubsection{Isolation and Independence}

A central principle in TLDR is the use of actors, based on the actor model. Actors are to be seen as entities with interaction. In other words, in order for a construct to qualify as an actor, it must define a way to behave when other actors interact with it. Actors should function independently, and in that regard, not be open to direct manipulation and only able to be changed through the messages it receives. This requirement is due to the wish of separation of processes, which will allow for greater concurrency by letting processes operate on local data instead of global, shared data. Therefore, TLDR tries to encourage natural isolation of functionality through actors, which in turn will also give a greater control of race conditions as no data is ever accessed by more than one process.

\subsubsection{The Main Actor}

Main is always the first actor, and any program writtin in TLDR is started with main receiving the arguments message. This is done to force the programmer to start and end the program with an actor, which will better support the actor modeling perspective. This means that the main actor is started with a message for the arguments, and when the main actor is killed the program will stop executing, whether or not there are still working actors.

The argument message called args, is a struct which consists of a argument counter called argv, and a lists of char lists called argv.

Another way that this affects the programmer, is the idea of spawning actors and having them send messages to main, instead of calling functions and having them return. This also means that there must be a way to reference main, since it is not spawned by another actor. This is solved by the introduction of the \enquote{me} keyword, which will evaluate to a reference to the current actor. This is very useful, especially if an actor wishes to delegate work to other actors. The message that is sent with the work, must simply contain a reference back to the delegating actor, which was included through the use of the \enquote{me} keyword.

Here are the semantics for the main actor.

\begin{align*}
&\inference[$\text{MAIN}$]{input \mapsto \Braket{S,\epsilon}}
                          {\Braket{\Treceive \Tr:args := \{S\},e} \Rightarrow_S \Braket{\Tr := input;S, e]}}
\end{align*}

\subsubsection{Construction of an Actor}
\label{sub:constructionOfAnActor}

The syntactical declaration of an actor is as follows:

\begin{lstlisting}
actor <identifier> := {
  <functionality>
}
\end{lstlisting}

And a concrete example could be:

\begin{verbatim}
  actor earth := {
    var temperature:real := 0;
    receive sunlight:light := {
      temperature := temperature + 0.1;
    }
  }
\end{verbatim}

As shown, the actor keyword precedes the definition, denoting the meaning of said definition. After the keyword an identifier of the declaration is needed, which will serve as the specific actor type. It is suggested that this identifier reflects the role of the actor in a context of use. Noticeably there is no \enquote{let} or \enquote{var} keyword in front of the definition, as there usually is when assigning. This is a deliberate choice since \enquote{let} and \enquote{var} implies interchangeability, which is not an option is this case. If variable declaration of actor definitions were possible, it would effectively be the equivalent of changing the definition of a type on run-time, which would make little sense, and completely undermine the type safety in the language.

Some general semantics of actors:

\begin{align*}
&\inference[TYPEOF]{}
                  {e \vdash m \Rightarrow_T t}
                  {, \mathbb{T}(m) = t}
\\\\
&\inference[$\text{ACTOR}$]{a' = a[\Tact \mapsto e \times st]}
                           {\Braket{\Tactor \Tact := \{S\}, a} \Rightarrow_S \Braket{S,e,at[\Tact \mapsto S],a'}}
\\\\
&\inference[$\text{ACTOR}$]{}
                           {\Braket{\Tactor \Tact := 1S, at} \Rightarrow_S \Braket{at[\Tact \mapsto S]}}
\end{align*}

\subsubsection{Basic Actor Functionality}
\label{subsubsec:BasicActorFunctionality}
There are four basic functionalities for actors: \enquote{spawn}, \enquote{die}, \enquote{send}, and \enquote{receive}, which are all used through keywords. It was desired to keep the syntax of these functionalities different from the syntax of functions. Even though they behave much like functions, taking input and giving output, they are more powerful. For example, regular functions cannot contain a type as a parameter, but the \enquote{spawn} functionality does this. Due to this and more differences, which follow below, it was decided to separate them syntactically.

The \enquote{spawn} functionality is used to create new instances of actors. And example could be:

\label{actorfuncSpawn}
\begin{lstlisting}
actor <identifier> := {
 <functionality>
}

let MyActor:<identifier> := spawn <identifier> <message>;

or alternatively:

var MyActor:<identifier> := spawn <identifier> <message>;
\end{lstlisting}

As can be seen above, there are four parts of the spawn functionality: an identifier, the keyword, the type of the actor, and optionally an initial message. Firstly, the identifier is preceded by a keyword for mutability, such as \enquote{let} or \enquote{var}. This allows for the substitution of handles, which provides possibilities of dynamic changes. This however opens up the possibility of \enquote{losing contact} with an actor, if the handle is replaced. This could potentially lead to memory leaks if not handled properly. \kanote{reference til garbage collection afsnit}

When spawning a new actor, you can also choose to add a message. The reason for this is to give the programmer a way of initialising the new actor with a certain message. In object-oriented languages this is usually done with a constructor, however doing it via a constructor would conflict with a central principle, since it would mean manipulating an actors state directly.

The semantics of \enquote{spawn} is as follows:

\begin{align*}
&\inference[$\text{SPAWN}$]{}
                       {\Braket{\Tlet \Tact:T := \Tspawn \; T \Tm,\Ta} \Rightarrow_S \Braket{\Tsend \Tact \Tm,\Ta[act \mapsto e \times st]}}
\\
&                       {, \Ta(x) \mapsto \Braket{S,p} , \Ta' = \Ta[\Tact \mapsto \Braket{S, p}]}
\end{align*}

Type rules for \enquote{spawn}:

\begin{align*}
\intertext{}
&\inference[SPAWN]{}
                 {\Tenv \mathbin{\text{spawn}} \; \Tt: \Tt}
\end{align*}

After an actor has been spawned, it will be possible to send messages to it. This is done with the \enquote{send}-keyword. This can be done as follows:

\label{actorfuncSend}
\begin{lstlisting}
MyMsg:int := 42;

Send MyActor MyMsg;
\end{lstlisting}

It is also possible for actors to send messages to themselves by using the \enquote{me}-keyword. Such messages will be treated the same as any other message.

The semantics of \enquote{send} is as follows:

\begin{align*}
&\inference[$\text{SEND}$]{e_2 = a(act),m \Rightarrow_T t}
                       {\Braket{\Tsend \Tact \Tm ; S,\Ta,e_1} \Rightarrow_S \Braket{S,e_1,a}|\Braket{\_t(m),e_2,a}}
\end{align*}

and the type rules for \enquote{send}:

\begin{align*}
\intertext{}
&\inference[SEND]{\Tenv m : \Tt & \Tenv a: \Tt'}
                 {\Tenv \mathbin{\text{send}} \; \mathbin{\text{m}} \; \mathbin{\text{a}} : ok }
\end{align*}

When an actor is sent a message, it must act according to a defined a way of handling that type of message. This definition is declared with the \enquote{receive}-keyword, which creates a method within the actor that is called when the actor receives a corresponding message. The syntax can be seen below:

\label{actorfuncReceive}
\begin{lstlisting}
actor <nameOfActor> := {
 receive <nameOfMessage>:<typeOfMessage> := {
  <functionality>
 }
}
\end{lstlisting}

In this example the receive-method defines the way messages of the type \enquote{<typeOfMessage>} are handled. Within the functionality \enquote{<nameOfMessage>} is the reference to the message. This message is immutable no matter if it was mutable where it was sent from. This is done to discourage further use of old messages.

It is also the intention to include a \enquote{wait on <typeOfMessage>} keyword, which will cause the actor to not evaluate the next message in the messagequeue, but instead traverse the queue, until a message matching \enquote{<typeOfMessage} is found. Then that message is de-queued and evaluated. This has not been implemented in the current release of TLDR, but it will be a central part of supporting discrete simulations.

The semantics of \enquote{receive} is as follows:

\begin{align*}
&\inference[$\text{RECEIVE}$]{}
                           {\Braket{\Treceive r:t := \{S\};,e} \Rightarrow_S \Braket{e[\_t \mapsto \Braket{S,r}]}}
\end{align*}

The type rules for \enquote{receive} are:

\begin{align*}
\intertext{}
&\inference[RECEIVE]{\Tenv m : \Tt}
                 {\Tenv \mathbin{\text{recieve}} \; \mathbin{\text{m}} :\Tt : ok }
\end{align*}

When an actor is no longer needed, it is possible to discard it with the \enquote{die}-keyword. In other languages, using the actor model, \enquote{die}, or similar functionality, is usually called by the parent of the actor, that is, the actor that spawned the actor. In TLDR however, \enquote{die} can only be called by the actor itself. This is done in order to keep the principle of only interacting with actors through messages, which is a simpler way of handling actors in TLDR, since the language does not have a built-in supervisor functionality as described in \cref{actSupervisors}. 

When an actor dies it stops immediately and does not compute further, and whatever messages might have been in the actors message-queue, will be lost. 

\subsubsection{Actor references}

In TLDR declarations must be accomodated by an assignment of a value, since no primitives can have a null value. However, since actors handles have the special position in the language of being the only value which is passed by reference, it should incompass a null value. The reason for this becomes apparent if one considers the following example:

\begin{lstlisting}
actor main := {
  receive arguments:args := {
    var jack:man := spawn man;
  }
}

actor man := {
  var bestMan:man := spawn man;
  }
}
\end{lstlisting}

In this example, the bestMan variable would result in an endless recursive spawn chain of actors. This is the most obvious problem, but there are multiple senarioes where forced value assignment for reference types become problematic. Due to this, the language allows for null references.

\subsubsection{Comparison}

When comparing actors, they are considered equal if they are reference equals.

Here are the type rules for comparison:

\begin{align*}
\intertext{}
&\inference[ACTOR]{\Tenv e_1: \Tt & \Tenv e_2: \Tt}
                 {\Tenv e_1 = e_2: \Tbool}                 
\end{align*}
