%mainfile: ../master.tex
\subsubsection{Static sematics}

Restrictions of the language that is hard or impossible to express in formal syntax, is analysed by formal sematics at compile time. This includes checking whether a identifier is used before its declaration, rendering it an illegal action by the formal sematics. 

The static sematic analysis phase includes following checking by the given sematics of TLDR:
\begin{itemize}
\item Type checking 
\item Function calls
\item Used before declared
\item H%iding checking
\item Scope checking
\end{itemize}

\emph{Type checking}
  Whether types of ide%ntifier makes sense to be operands in operation
  
\emph{Function calls}
  Right type of identifier and correct numbe%r of parameters being applied to functions calls 

\emph{Used before declared}
  %When is identifier declared according to use
  
  
  
\emph{Hiding checking}
  %If similar named identifiers are used in illegal manor, whithin same scopes
  
  This is done by looping thru every entry of the systemtable. For every identifier, check if theres is another identifier with a matching name. If such case exist check if either one of them is visible in scopes to the other

\emph{Scope checking}
  Is identifier visible in this scope
