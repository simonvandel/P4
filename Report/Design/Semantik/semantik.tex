\section{Semantics}
This section describes the semantics of the constructs in TLDR both formally and informally. It is split into pieces each describing one construct both formally and informally.

\subsection{Formal semantics in TLDR}
To describe the semantics formally it was decided to use small-step semantics. The alternative was big-step semantics but since the language depends heavily on non-determinism and parallelism and this is not possible to represent in big-step semantics, small-step semantic was a better choice.

The formal rules have three parts as illustrated in \cref{SS-semantics}

\begin{figure}[H]
\begin{align*}
&\inference[$RULE-NAME$]{[PREMISE]}
												{[CONCLUSSION]}
												{,[SIDE-CONDITION]}
\end{align*}
\caption{A desription of formal small-step semantics}
\label{SS-semantics}
\end{figure}

\subsection{Actor-Environment model}
The formal semantics will also use an actor-environment model.

$a = \text{Anames} \cup \{next\} \rightharpoonup e \times st$

$e = \text{Symbols} \rightharpoonup \text{Stm} \times \text{Symbols}$
$st = \text{Structs} \rightharpoonup \text{Stm}$
$at = \text{ActorTypes} \rightharpoonup \text{Stm}$

\subsection{General Constructs}
\begin{align*}
&\inference[$\text{INVOKE}$]{\Braket{S_1,e[p_1' \mapsto \Braket{S_2,p_2'}]} \Rightarrow_S \Braket{S_1',e'}}
                  {\Braket{p_1(p_2),e} \Rightarrow_S \Braket{S_1',e'}}
\\
&									{,e(p_1) = \Braket{S_1,p_1'}, e(p_2) = \Braket{S_2,p_2'}}
\end{align*}





\subsection{Statements}

\subsubsection{General}
\begin{align*}
%&\inference[$\text{DECL}$]{}
%                         {\Braket{\Tlet x := a,e,st} \Rightarrow_S \Braket{x := a, e',st}}
%												{, l = e(next), e' = e[x\mapsto l, next\mapsto new(l)]}
%\\\\
&\inference[$\text{INIT}_{SYM}$]{}
                         {\Braket{\Tlet \Tx := \{S\};,e} \Rightarrow_S e'}
												 {, e' = e[\Tx \mapsto \Braket{S,\epsilon}]}
\\\\
&\inference[$\text{INIT}_{FUNC}$]{}
                         {\Braket{\Tlet \Tx(y) := \{S\};,e} \Rightarrow_S e'}
												 {, e' = e[\Tx \mapsto \Braket{S,y}]}
\\\\
&\inference[$\text{INIT}$]{}
                         {\Braket{\Tvar \Tx := S,e} \Rightarrow_S e'}
												 {, e' = e[\Tx \mapsto \Braket{S,\epsilon}]}
\\\\
&\inference[ASS]{}
                 {\Braket{\Tx := S,e} \Rightarrow_S e'}
								 {, e' = e[\Tx \mapsto \Braket{S,\epsilon}]}
\\\\
&\inference[$\text{STATEMENTS}_1$]{\Braket{S_1,e} \Rightarrow_S \Braket{S_1',e'}}
                         {\Braket{S_1;S_2,e} \Rightarrow_S \Braket{S_1';S_2,e'}}
\\\\
&\inference[$\text{STATEMENTS}_2$]{\Braket{S_1,e} \Rightarrow_S e'}
                         {\Braket{S_1;S_2,e} \Rightarrow_S \Braket{S_2,e'}}
\\\\
&\inference[$\text{BLOCK}_1$]{\Braket{S,e} \Rightarrow_S \Braket{S',e'}}
                         {\Braket{\{S\},e} \Rightarrow_S \Braket{\{S'\},e'}}
\\\\
&\inference[$\text{BLOCK}_1$]{\Braket{S,e} \Rightarrow_S e'}
                         {\Braket{\{S\},e} \Rightarrow_S e'}
\end{align*}


\subsubsection{Misc}

\begin{align*}
&\inference[TYPEOF]{}
                  {e \vdash m \Rightarrow_T t}
									{, \mathbb{T}(m) = t}
\\\\
&\inference[$\text{ACTOR}$]{a' = a[\Tact \mapsto e \times st]}
                           {\Braket{\Tactor \Tact := \{S\}, a} \Rightarrow_S \Braket{S,e,at[\Tact \mapsto S],a'}}
\\\\
&\inference[$\text{ACTOR}$]{}
                           {\Braket{\Tactor \Tact := 1S, at} \Rightarrow_S \Braket{at[\Tact \mapsto S]}}
\\\\
&\inference[$\text{RECIEVE}$]{}
                           {\Braket{\Treceive r:t := \{S\};,e} \Rightarrow_S \Braket{e[\_t \mapsto \Braket{S,r}]}}
%%%%%%%%%%%%%%%%%%%%%%%%%%%%%%%%%%%%%%%%%%%%%%%%%%%%%%%%%%%%%%%
\intertext{The left side of a parallel statement is executed, but not finished. The environment and actor model is updated when executing $S_1$.}
&\inference[$\text{PAR}_1$]{\Braket{S_1,e_1,\Ta} \Rightarrow_S \Braket{S_1',e_1',\Ta'}} 
                           {\Braket{S_1,e_1,\Ta}|\Braket{S_2,e_2,\Ta} \Rightarrow_S \Braket{S_1',e_1',\Ta'}|\Braket{S_2,e_2,\Ta'}}
%%%%%%%%%%%%%%%%%%%%%%%%%%%%%%%%%%%%%%%%%%%%%%%%%%%%%%%%%%%%%%%
\intertext{The left side of a parallel statement is executed, and finishes. The environment and actor model is updated when executing $S_1$.}
&\inference[$\text{PAR}_2$]{\Braket{S_1,e_1,\Ta} \Rightarrow_S \Braket{e_1',\Ta'}} 
                           {\Braket{S_1,e_1,\Ta}|\Braket{S_2,e_2,\Ta} \Rightarrow_S \Braket{S_2,e_2,\Ta'}}
%%%%%%%%%%%%%%%%%%%%%%%%%%%%%%%%%%%%%%%%%%%%%%%%%%%%%%%%%%%%%%%
\intertext{The right side of a parallel statement is executed, but not finished. The environment and actor model is updated when executing $S_2$.}
&\inference[$\text{PAR}_3$]{\Braket{S_2,e_1,\Ta} \Rightarrow_S \Braket{S_2',e_1',\Ta'}} 
                           {\Braket{S_1,e_1,\Ta}|\Braket{S_2,e_2,\Ta} \Rightarrow_S \Braket{S_1',e_1',\Ta'}|\Braket{S_2,e_2,\Ta'}}
%%%%%%%%%%%%%%%%%%%%%%%%%%%%%%%%%%%%%%%%%%%%%%%%%%%%%%%%%%%%%%%
\intertext{The right side of a parallel statement is executed, and finishes. The environment and actor model is updated when executing $S_2$.}
&\inference[$\text{PAR}_4$]{\Braket{S_2,e_1} \Rightarrow_S \Braket{e_1',\Ta'}}
                           {\Braket{S_1,e_1,\Ta}|\Braket{S_2,e_2,\Ta} \Rightarrow_S \Braket{S_1',e_1',\Ta'}}
%%%%%%%%%%%%%%%%%%%%%%%%%%%%%%%%%%%%%%%%%%%%%%%%%%%%%%%%%%%%%%%
\\\\
&\inference[$\text{SPAWN}$]{}
                       {\Braket{\Tlet \Tact:T := \Tspawn \; T \Tm,\Ta} \Rightarrow_S \Braket{\Tsend \Tact \Tm,\Ta[act \mapsto e \times st]}}
\\
&												{, \Ta(x) \mapsto \Braket{S,p} , \Ta' = \Ta[\Tact \mapsto \Braket{S, p}]}
\\\\
&\inference[$\text{SEND}$]{e_2 = a(act),m \Rightarrow_T t}
                       {\Braket{\Tsend \Tact \Tm ; S,\Ta,e_1} \Rightarrow_S \Braket{S,e_1,a}|\Braket{\_t(m),e_2,a}}
\\\\
&\inference[$\text{MAIN}$]{input \mapsto \Braket{S,\epsilon}}
                          {\Braket{\Treceive \Tr:args := \{S\};,e} \Rightarrow_S \Braket{\Tr := input;S, e]}}
\end{align*}
