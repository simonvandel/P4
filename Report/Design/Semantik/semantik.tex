\section{Semantics}
This section describes the semantics of the constructs in TLDR both formally and informally. It is split into pieces each describing one construct both formally and informally.

\subsection{Formal semantics in TLDR}
To describe the semantics formally it was decided to use small-step semantics. The alternative was big-step semantics but since the language depends heavily on non-determinism and parallelism and this is not possible to represent in big-step semantics, small-step semantic was a better choice.

The formal rules have three parts as illustrated in \cref{SS-semantics}

\begin{figure}[H]
\begin{align*}
&\inference[$RULE-NAME$]{[PREMIS]}
												{[CONCLUSSION]}
												{,[SIDE-CONDITION]}
\end{align*}
\caption{A desription of formal small-step semantics}
\label{SS-semantics}
\end{figure}

\subsection{Actor-Environment model}
The formal semantics will also use an actor-environment model.

$a = \text{Anames} \cup \{next\} \rightharpoonup e$

$e = \text{Symbols} \rightharpoonup \text{Stm} \times \text{Symbols}$

\subsection{General Constructs}
\begin{align*}
&\inference[$\text{INVOKE}$]{\Braket{S_1,e[p_1' \mapsto \Braket{S_2,p_2'}]} \Rightarrow_S \Braket{S_1',e'}}
                  {\Braket{p_1(p_2),e} \Rightarrow_S \Braket{S_1',e'}}
\\
&									{,e(p_1) = \Braket{S_1,p_1'}, e(p_2) = \Braket{S_2,p_2'}}
\end{align*}



\subsection{Boolean Expressions}
\newcommand{\Tand}{\text{\; AND \;}}
\newcommand{\Tnot}{\text{NOT \;}}
\newcommand{\Tnand}{\text{\; NAND \;}}
\newcommand{\Tor}{\text{\; OR \;}} 
\newcommand{\Tnor}{\text{\; NOR \;}}
\newcommand{\Txor}{\text{\; XOR \;}}

\begin{align*}
&\inference[$\text{INVOKE}_{B\top}$]{\Braket{S_1,e[p_1' \mapsto \Braket{S_2,p_2'}]} \Rightarrow_B \top}
                  {\Braket{p_1(p_2),e} \Rightarrow_B \top}
\\									
&									{,e(p_1) = \Braket{S_1,p_1'}, e(p_2) = \Braket{S_2,p_2'}}
\\\\
&\inference[$\text{INVOKE}_{B\bot}$]{\Braket{S_1,e[p_1' \mapsto \Braket{S_2,p_2'}]} \Rightarrow_B \bot}
                  {\Braket{p_1(p_2),e} \Rightarrow_B \bot}
\\									
&									{,e(p_1) = \Braket{S_1,p_1'}, e(p_2) = \Braket{S_2,p_2'}}
\\\\
&\inference[$\text{PARENS}_\text{B}$]{e \vdash b_1 \Rightarrow_B b_1'}
                       {e \vdash (b_1) \Rightarrow_B (b_1')}
\\\\
&\inference[$\text{NOT}_\top$]{e \vdash b_1 \Rightarrow_B \top}
                       {e \vdash \Tnot b_1 \Rightarrow_B \bot}
\\\\
&\inference[$\text{NOT}_\bot$]{e \vdash b_1 \Rightarrow_B \bot}
                       {e \vdash \Tnot b_1 \Rightarrow_B \top}
\\\\
&\inference[$\text{AND}_1$]{e \vdash b_1 \Rightarrow_B \bot}
                    {e \vdash b_1 \Tand b_2 \Rightarrow_B \bot}
\\\\
&\inference[$\text{AND}_2$]{e \vdash b_1 \Rightarrow_B \top \\ b_2 \Rightarrow_B \bot}
								    {e \vdash b_1 \Tand b_2 \Rightarrow_B \bot}
\\\\
&\inference[$\text{AND}_3$]{e \vdash b_1 \Rightarrow_B \top \\ b_2 \Rightarrow_B \top}
								    {e \vdash b_1 \Tand b_2 \Rightarrow_B \top}
\\\\
&\inference[NAND]{}
								   {e \vdash b_1 \Tnand b_2 \Rightarrow_B \Tnot( b_1 \Tand b_2 )}
\\\\
&\inference[OR]{}
                 {e \vdash b_1 \Tor b_2 \Rightarrow_B \Tnot(\Tnot b_1 \Tand \Tnot b_2)}
\\\\
&\inference[NOR]{}
								   {e \vdash b_1 \Tnor b_2 \Rightarrow_B \Tnot( b_1 \Tor b_2 )}
\\\\
&\inference[XOR]{}
                  {e \vdash b_1 \Txor b_2 \Rightarrow_B (\Tnot(b_1 \Tand b_2)) \Tand (b_1 \Tor b_2)}
\end{align*}

\subsection{Logical Expressions}
\begin{align*}
&\inference[$\text{EQUALS}_\text{L}$]{e \vdash a_1 \Rightarrow_B a_1'}
                    {e \vdash a_1 = a_2 \Rightarrow_B a_1' = a_2}
&
&\inference[$\text{EQUALS}_\text{R}$]{e \vdash a_1 \Rightarrow_B a_1'}
                    {e \vdash a_2 = a_1 \Rightarrow_B a_2 = a_1'}
\\\\
&\inference[$\text{EQUALS}_\text{V1}$]{}
                    {v_1 = v_2 \Rightarrow_B \top}
										{, v_1 = v_2}
&
&\inference[$\text{EQUALS}_\text{V2}$]{}
                    {v_1 = v_2 \Rightarrow_B \bot}
										{, v_1 \neq v_2}
\\\\
&\inference[$\text{EQUALS}_\text{Actor}$]{a(act_1) = e\\ a(act_2) = e}
                    {act_1 = act_2 \Rightarrow_B \top}
&
&\inference[$\text{EQUALS}_\text{Actor}$]{a(act_1) = e\\ a(act_2) = e'}
                    {act_1 = act_2 \Rightarrow_B \bot}
										{e \neq e'}
\\\\
&\inference[$NEQUALS$]{}
                    {e \vdash a_1 != a_2 \Rightarrow_B \Tnot (a_1 = a_2)}
\\\\
&\inference[$\text{GT}_\text{L}$]{e \vdash a_1 \Rightarrow_A a_1'}
                    {e \vdash a_1 > a_2 \Rightarrow_A a_1' > a_2}
&
&\inference[$\text{GT}_\text{R}$]{e \vdash a_1 \Rightarrow_A a_1'}
                    {e \vdash a_2 > a_1 \Rightarrow_A a_2 > a_1'}
\\\\
&\inference[$\text{GT}_\text{V1}$]{}
                    {v_1 > v_2 \Rightarrow_B \top}
										{, v_1 > v_2}
&
&\inference[$\text{GT}_\text{V2}$]{}
                    {v_1 > v_2 \Rightarrow_B \bot}
										{, v_1 \leq v_2}
\\\\
&\inference[$GTEQ$]{}
                    {e \vdash a_1 >= a_2 \Rightarrow_A (a_1 > a_2) \Tor (a_1 = a_2)}
\\\\
&\inference[$\text{LT}_\text{L}$]{e \vdash a_1 \Rightarrow_A a_1'}
                    {e \vdash a_1 < a_2 \Rightarrow_A a_1' < a_2}
&
&\inference[$\text{LT}_\text{R}$]{e \vdash a_1 \Rightarrow_A a_1'}
                    {e \vdash a_2 < a_1 \Rightarrow_A a_2 < a_1'}
\\\\
&\inference[$\text{LT}_\text{V1}$]{}
                    {v_1 < v_2 \Rightarrow_B \top}
										{, v_1 < v_2}
&
&\inference[$\text{LT}_\text{V2}$]{}
                    {v_1 < v_2 \Rightarrow_B \bot}
										{, v_1 \geq v_2}
\\\\
&\inference[$LTEQ$]{}
                    {e \vdash a_1 <= a_2 \Rightarrow_A (a_1 < a_2) \Tor (a_1 = a_2)}
\end{align*}
\subsection{Statements}
\newcommand{\Tx}{\mathbin{\; \text{x} \;}}
\newcommand{\Tlet}{\text{let}}
\newcommand{\Tvar}{\text{var}}
\subsubsection{General}
\begin{align*}
%&\inference[$\text{DECL}$]{}
%                         {\Braket{\Tlet x := a,e,st} \Rightarrow_S \Braket{x := a, e',st}}
%												{, l = e(next), e' = e[x\mapsto l, next\mapsto new(l)]}
%\\\\
&\inference[$\text{INIT}_{SYM}$]{}
                         {\Braket{\Tlet \Tx := S,e} \Rightarrow_S e'}
												 {, e' = e[\Tx \mapsto \Braket{S,\epsilon}]}
\\\\
&\inference[$\text{INIT}_{FUNC}$]{}
                         {\Braket{\Tlet \Tx(y) := S,e} \Rightarrow_S e'}
												 {, e' = e[\Tx \mapsto \Braket{S,y}]}
\\\\
&\inference[$\text{INIT}$]{}
                         {\Braket{\Tvar \Tx := S,e} \Rightarrow_S e'}
												 {, e' = e[\Tx \mapsto \Braket{S,\epsilon}]}
\\\\
&\inference[ASS]{}
                 {\Braket{\Tx := S,e} \Rightarrow_S e'}
								 {, e' = e[\Tx \mapsto \Braket{S,\epsilon}]}
\\\\
&\inference[$\text{STATEMENTS}_1$]{\Braket{S_1,e} \Rightarrow_S \Braket{S_1',e'}}
                         {\Braket{S_1;S_2,e} \Rightarrow_S \Braket{S_1';S_2,e'}}
\\\\
&\inference[$\text{STATEMENTS}_2$]{\Braket{S_1,e} \Rightarrow_S e'}
                         {\Braket{S_1;S_2,e} \Rightarrow_S \Braket{S_2,e'}}
\\\\
&\inference[$\text{BLOCK}_1$]{\Braket{S,e} \Rightarrow_S \Braket{S',e'}}
                         {\Braket{\{S\},e} \Rightarrow_S \Braket{\{S'\},e'}}
\\\\
&\inference[$\text{BLOCK}_1$]{\Braket{S,e} \Rightarrow_S e'}
                         {\Braket{\{S\},e} \Rightarrow_S e'}
\end{align*}

\subsubsection{Conditional}
\newcommand{\Tif}{\text{if}}
\newcommand{\Telse}{\text{else}}

\begin{align*}
&\inference[$\text{IF}_\top$]{}
                      {\Braket{if(b)\{S\},e} \Rightarrow_S \Braket{\{S\},e}}
											{,b \Rightarrow_B \top}
\\\\
&\inference[$\text{IF}_\bot$]{}
                      {\Braket{if(b)\{S\},e} \Rightarrow_S e}
											{,b \Rightarrow_B \bot}
\\\\
&\inference[$\text{IF-ELSE}_\top$]{}
                      {\Braket{if(b)\{S_1\}else\{S_2\},e} \Rightarrow_S \Braket{\{S_1\},e}}
											{,b \Rightarrow_B \top}
\\\\
&\inference[$\text{IF-ELSE}_\bot$]{}
                      {\Braket{if(b)\{S_1\}else\{S_2\},e} \Rightarrow_S \Braket{\{S_2\},e}}
											{,b \Rightarrow_B \bot}
\end{align*}
\subsubsection{Loop}
\newcommand{\Tfor}{\mathbin{\text{for}}}
\newcommand{\Tin}{\mathbin{\text{in}}}
\newcommand{\Twhile}{\mathbin{\text{while}}}

\begin{align*}
&\inference[$\text{WHILE}_\top$]{e \vdash b \Rightarrow_B \top}
                       {\Braket{\Twhile(b)\{S\},e} \Rightarrow_S \Braket{\{S\}; \Twhile (b)\{S\},e}}
\\\\
&\inference[$\text{WHILE}_\bot$]{e \vdash b \Rightarrow_B \bot}
                       {\Braket{\Twhile(b)\{S\},e} \Rightarrow_S e}
\\\\
&\inference[$\text{FOR}_{1}$]{}
                       {\Braket{\Tfor \Tx \Tin [n_1,\dots,n_m]\{S\},e} \Rightarrow_S \Braket{x := n_1;\{S\}; \Tfor \mathbin{\text{x}} \mathbin{\text{in}} [n_2,\dots,n_m]\{S\},e}}
\\\\
&\inference[$\text{FOR}_{2}$]{}
                       {\Braket{\Tfor \mathbin{\text{x}} \mathbin{\text{in}} [n_m]\{S\},e} \Rightarrow_S \Braket{x := n_m;\{S\},e}}
\end{align*}

\subsubsection{Misc}
\newcommand{\Tspawn}{\mathbin{\text{spawn}}}
\newcommand{\Tsend}{\mathbin{\text{send}}}
\newcommand{\Ta}{\; \mathbin{\text{a}} \;}
\newcommand{\Tm}{\; \mathbin{\text{m}} \;}
\newcommand{\Twhere}{\mathbin{\text{where}}}
\newcommand{\Ten}{\mathbin{\text{env}}}
\newcommand{\Tactor}{\mathbin{\text{actor}} \;}
\newcommand{\Tact}{\; \mathbin{\text{act}} \;}
\newcommand{\Trecieve}{\mathbin{\text{recieve}} \;}
\newcommand{\Tr}{\; \mathbin{\text{r}} \;}

\begin{align*}
&\inference[TYPEOF]{}
                  {e \vdash m \Rightarrow_T t}
									{, \mathbb{T}(n) = t}
\\\\
&\inference[$\text{ACTOR}$]{a' = a[\Tact \mapsto e]}
                           {\Braket{\Tactor \Tact := \{S\};,a} \Rightarrow_S \Braket{S,e,a'}}
\\\\
&\inference[$\text{RECIEVE}$]{}
                           {\Braket{\Trecieve r:t := \{S\};,e} \Rightarrow_S \Braket{e[\_t \mapsto \Braket{S,r}]}}
%%%%%%%%%%%%%%%%%%%%%%%%%%%%%%%%%%%%%%%%%%%%%%%%%%%%%%%%%%%%%%%
\intertext{The left side of a parallel statement is executed, but not finished. The environment and actor model is updated when executing $S_1$.}
&\inference[$\text{PAR}_1$]{\Braket{S_1,e_1,\alpha} \Rightarrow_S \Braket{S_1',e_1',\alpha'}} 
                           {\Braket{S_1,e_1,\alpha}|\Braket{S_2,e_2,\alpha} \Rightarrow_S \Braket{S_1',e_1',\alpha'}|\Braket{S_2,e_2,\alpha'}}
%%%%%%%%%%%%%%%%%%%%%%%%%%%%%%%%%%%%%%%%%%%%%%%%%%%%%%%%%%%%%%%
\intertext{The left side of a parallel statement is executed, and finishes. The environment and actor model is updated when executing $S_1$.}
&\inference[$\text{PAR}_2$]{\Braket{S_1,e_1,\alpha} \Rightarrow_S \Braket{e_1',\alpha'}} 
                           {\Braket{S_1,e_1,\alpha}|\Braket{S_2,e_2,\alpha} \Rightarrow_S \Braket{S_2,e_2,\alpha'}}
%%%%%%%%%%%%%%%%%%%%%%%%%%%%%%%%%%%%%%%%%%%%%%%%%%%%%%%%%%%%%%%
\intertext{The right side of a parallel statement is executed, but not finished. The environment and actor model is updated when executing $S_2$.}
&\inference[$\text{PAR}_3$]{\Braket{S_2,e_1,\alpha} \Rightarrow_S \Braket{S_2',e_1',\alpha'}} 
                           {\Braket{S_1,e_1,\alpha}|\Braket{S_2,e_2,\alpha} \Rightarrow_S \Braket{S_1',e_1',\alpha'}|\Braket{S_2,e_2,\alpha'}}
%%%%%%%%%%%%%%%%%%%%%%%%%%%%%%%%%%%%%%%%%%%%%%%%%%%%%%%%%%%%%%%
\intertext{The right side of a parallel statement is executed, and finishes. The environment and actor model is updated when executing $S_2$.}
&\inference[$\text{PAR}_4$]{\Braket{S_2,e_1} \Rightarrow_S \Braket{e_1',\alpha'}}
                           {\Braket{S_1,e_1,\alpha}|\Braket{S_2,e_2,\alpha} \Rightarrow_S \Braket{S_1',e_1',\alpha'}}
%%%%%%%%%%%%%%%%%%%%%%%%%%%%%%%%%%%%%%%%%%%%%%%%%%%%%%%%%%%%%%%
\\\\
&\inference[$\text{SPAWN}$]{}
                       {\Braket{\Ta := \Tspawn \Tx \Tm,\alpha} \Rightarrow_S \Braket{\Tsend \Ta \Tm,\alpha'}}
\\
&												{, \alpha(x) \mapsto \Braket{S,p} , \alpha' = \alpha[\Ta \mapsto \Braket{S, p}]}
\\\\
&\inference[$\text{SEND}$]{e_2 = a(act),m \Rightarrow_T t}
                       {\Braket{\Tsend \Tact \Tm ; S,\alpha,e_1} \Rightarrow_S \Braket{S,e_1,a}|\Braket{\_t(m),e_2,a}}
\\\\
&\inference[$\text{MAIN}$]{input \mapsto \Braket{S,\epsilon}}
                          {\Braket{\Trecieve \Tr:args := \{S\};,e} \Rightarrow_S \Braket{\Tr := input;S, e]}}
\end{align*}
