\section{Compiler Compiler Choice}
\label{sec:compiler_compiler_choice}

It was decided to prioritize the programmers above a simple grammar. This means that keeping the chosen syntax is preferred, and therefore accepting a more complex grammar where it should needed. Due to this possibility of complexity it was chosen to use a compiler-compiler. The choice of compiler-compiler was decided based on a series of criteria, which are listed below with the corresponding reason for the criteria.

\begin{enumerate}
\item \textbf{Language of the compiler} It was chosen to construct the compiler in a language compatible with the common language infrastructure (CLI), since the developers had previous experience with this. This makes C\# a preferred target language for the compiler compiler as well as F\#.

\item \textbf{LR parser} The compiler-compiler should support LR parsing because left recursions has been used in the grammar specifying the language.\\

\item \textbf{Action Code} The parser should support action code and grammar and action code should be in separated files.\\

\end{enumerate}

Two compiler-compilers matching the criteria list were found. These were HIME and sableCC. sabelCC supports LALR parsing, which parses the same set of grammars as LR parser, which makes sabelCC a suitable parser. HIME has been choses as the compiler-compiler because it was latest updated and targets C\# as the main output language where SabelCC targets java and later on started supporting C\#.