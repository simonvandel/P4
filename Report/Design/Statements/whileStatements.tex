%mainfile: ../master.tex
\subsection{While-loop}
\label{subsec:whileLoopStatements}

The while-loop statement runs a block of statements until a boolean expression provided returns false.

\subsubsection{Syntax}

\begin{grammar}
<While> ::= 'while' <Expression> <Block>
\end{grammar}

A concrete example:

\begin{verbatim}
  var i:int := 0;
  while (i < 10) {i := i + 1}
\end{verbatim}

\subsubsection{Semantics}

\begin{align*}
\intertext{In the case that the boolean expression $b$ is true, the rule for composition of statements is used. First, the statement $S$ is evaluated and then the while statement is run again. The statement $S$ may have changed $b$.}
&\inference[$\text{WHILE}_\top$]{sEnv \vdash b \Rightarrow_B \top}
                       {\Braket{\Twhile(b)\{S\},sEnv} \Rightarrow_S \Braket{\{S\}; \Twhile (b)\{S\},sEnv}}
\\\\
\intertext{In the case that the boolean expression $b$ is false, the while statement has ended, and can simply be rewritten to the environment $sEnv$}
&\inference[$\text{WHILE}_\bot$]{sEnv \vdash b \Rightarrow_B \bot}
                       {\Braket{\Twhile(b)\{S\},sEnv} \Rightarrow_S sEnv}
\end{align*}

\subsubsection{Type Rules}

\begin{align*}
\intertext{The conditional body of the while construct must be of type bool. The body of the while-loop can be of any type defined in $\Tt$}
&\inference[WHILE]{\Tenv b : \Tbool &
                  \Tenv e : \Tt}
                 {\Tenv \mathbin{\text{while}} \; (b) \; {e}: ok}
\end{align*}
