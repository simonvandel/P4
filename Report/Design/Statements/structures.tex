%mainfile: ../../master.tex
\subsection{Structures}
\label{subsec:structs}

In TLDR there are three ways to do encapsulation. Actors, Tuples and structs. Structs are unique by being fully accessible within the scope, and having named fields. Structs are especially useful in TLDR for creating messages.

\subsubsection{Defining Structures}
\label{sec:defStructures}

\subsubsection{Syntax}

Structures are defined by using the \enquote{struct} keyword. The grammar for declaring structs are as follows:

\begin{grammar}
  <Struct> ::= 'struct' <Identifier> ':= \{' <TypeDecls> '\}'
\end{grammar}

And a concrete example:

\begin{lstlisting}[style=TLDR]
  struct Person := {Name:[char]; Age:int}
\end{lstlisting}

\subsubsection{Semantics}

having these semantics:

\begin{align*}
\intertext{In the case that we have multiple S of assignments, we can rewrite the T to now map to new s' that includes x, in the new st' environment and the rest of declaration statements}
&\inference[$\text{STRUCT}$]{}
                            {\Braket{\Tstruct T := \{x:T';S\}, env_s, st} \Rightarrow_S \Braket{\Tstruct T := \{S\},env_s,st'}}
\\
&{\Twhere st' = st[T \mapsto s'],st(T) = s,s' = s \cup x]}
\\\\
\intertext{In the case that we have multiple S of assignments, we can rewrite the T to now map to new s' that includes x, in the new st' environment}
&\inference[$\text{STRUCT}$]{}
                            {\Braket{\Tstruct T := \{x:T\}, env_s, st} \Rightarrow_S \Braket{env_s,st'}}
\\
&{\Twhere st' = st[T \mapsto s'],st(T) = s,s' = s \cup x]}
\end{align*}

\begin{align*}
&\inference[$\text{STRUCT}$]{}
                            {\Braket{\Tstruct T_1 := \{x:T_2;S\}, env_s} \Rightarrow_S \Braket{\Tstruct T_1 := \{S\},env_s'}}
\\
&{\Twhere env_s' = env_s[T \mapsto \Braket{\epsilon,s'}],env_s(T) = s,s' = s \cup x]}
\\\\
&\inference[$\text{STRUCT}$]{}
                            {\Braket{\Tlet \; x:T := (S),env_s} \Rightarrow_S \Braket{x:T := (S),env_s'}}
\\
&{env_s' = env_s[x \mapsto (S,s)],env_s(T) = (\epsilon,s)]}
\\\\
&\inference[$\text{STRUCT}$]{}
                            {\Braket{s:T := (f := x;S),env_s} \Rightarrow_S \Braket{s.f := x;s:T := (S),env_s}}
\end{align*}

\subsubsection{Type Rules}

\begin{align*}
\intertext{The elements can be of any type defined in $\Tt$}
&\inference[STRUCT]{E[s \mapsto (e_1:\Tt_1;e_2:\Tt_2;...;e_n:\Tt_n) \rightarrow ok]\vdash S : ok & }
                 {\Tenv \mathbin{\text{struct s}} := \{e_1:\Tt_1;e_2:\Tt_2;...;e_n:\Tt_n\}; S: ok}
\end{align*}



\subsubsection{Initialising Structures}
\label{sec:initStructures}

Structs can be initialised and assigned to symbols using either a constant assignment or a variable assignment. Structs initialised as a constant assignment cannot change any of the fields of the structs; the struct is immutable. Structs initialised as a variable assignment can change all of its fields at any time; the struct is mutable.

\subsubsection{Syntax}

The syntax for initialising a struct is as follow.

\begin{grammar}
<StructLiteral> ::= '(' (<Reassignment>';')* ')' (':' <Identifier>)?
\end{grammar}


With concrete examples for immutables:

\begin{verbatim}
  let Alice:Person := (Name := "Alice"; Age := 20);
\end{verbatim}

And for mutable struct is as follow.

\begin{verbatim}
  var Alice:Person := (Name := "Alice"; Age := 20);
\end{verbatim}

And for usage in lists.

\begin{verbatim}
  [(Name := "Alice"; Age := 20):Person];
\end{verbatim}

\subsubsection{Access to Structure Fields}
\label{sec:accessStructFields}

Fields can be access using the following syntax.

\begin{verbatim}
  Alice.Name; // "Alice"
\end{verbatim}

Structs declared as mutable can have fields reassigned using the following syntax.

\begin{verbatim}
  Alice.Name; // "Alice"
  Alice.Name := "Bob";
  Alice.Name; // "Bob";
\end{verbatim}
 
\subsubsection{Semantics}

\begin{align*}
\intertext{In the case that we have multiple S of assignments, we can rewrite the s to now having an accessor that maps to value of the first assignment and the rest of pending assignments in s}
&\inference[$\text{STRUCT}$]{}
                            {\Braket{\Tlet \; s:T := (f := x;S),env_s,st} \Rightarrow_S \Braket{s.f := x;\Tlet \; s:T := (S),env_s,st}}
\\\\
\intertext{In the case that we have only one assignment, we can rewrite the s to now having an accessor that maps to value of the assignment}
&\inference[$\text{STRUCT}$]{}
                            {\Braket{\Tlet \; s:T := (f := x),env_s,st} \Rightarrow_S \Braket{s.f := x,env_s,st}}
%\\\\
%&\inference[$\text{STRUCT}$]{}
%                            {\Braket{(f := x;S),env_s,st} \Rightarrow_S \Braket{s.f := x;\Tlet \; s:T := (S),env_s,st}}
%\\\\
%&\inference[$\text{STRUCT}$]{}
%                            {\Braket{(f := x):T,env_s,st} \Rightarrow_S \Braket{s.f := x,env_s,st}}
\end{align*}

\subsubsection{Type Rules}

Each e of type t is matching the declared struct types, evaluates to the type of the declared struct.

\begin{align*}
&\inference[STRUCTLITERAL]{\Tenv (e_1 : \Tt_1;e_2 : \Tt_2;...;e_n : \Tt_n):\Tt'}
                 {\Tenv \mathbin{\text{(}} e_1; e_2;...;e_n\mathbin{\text{)}}:\Tt': \Tt'}
\end{align*}

\subsubsection{Comparison of structs}
\begin{align*}
\intertext{}
&\inference[STRUCT]{\Tenv e_1: \Tt & \Tenv e_2: \Tt}
                 {\Tenv e_1 = e_2: \Tbool}
\end{align*}
