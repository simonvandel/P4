%mainfile: ../master.tex
\subsection{If-statements}
\label{subsec:ifStatements}

If-statements can be written either as a if-else statement or just as an if-statement. In an if-else statement, the body just after the condition is evaluated if the condition evaluates to the boolean value \emph{true}. If the condition evaluates to \emph{false}, the body after the else keyword is evaluated. In an if-statement, the body after the condition is run, if the condition has the value \emph{true}. If the condition has value \emph{false}, nothing is evaluated.

\subsubsection{Syntax}

\begin{grammar}
<If> ::= 'if' <Expression> <Block>

<IfElse> ::= 'if' <Expression> <Block> 'else' <Block>
\end{grammar}

A concrete example:

\begin{verbatim}
  // if-then-else statement
  if (2 + 2 = 4) {"math works!"}
  else {"something is wrong here!"}

  // if-statement
  if (remainingTime < 10) {initiateCountdown()}
\end{verbatim}

\subsubsection{Semantics}

\begin{align*}
\intertext{In the case that the condition $b$ is true, the rule for the if-statement can simply be rewritten to the statement $S$.}
&\inference[$\text{IF}_\top$]{}
                      {\Braket{if(b)\{S\},sEnv} \Rightarrow_S \Braket{\{S\},sEnv}}
                      {,b \Rightarrow_B \top}
\\\\
\intertext{In the case that the condition $b$ is false, nothing is evaluated, and the if-statement is rewritten to the symbol environment $sEnv$.}
&\inference[$\text{IF}_\bot$]{}
                      {\Braket{if(b)\{S\},sEnv} \Rightarrow_S sEnv}
                      {,b \Rightarrow_B \bot}
\\\\
\intertext{In the case that the condition $b$ is true, the rule for the if-else statement is rewritten to $S_1$.}
&\inference[$\text{IF-ELSE}_\top$]{}
                      {\Braket{if(b)\{S_1\}else\{S_2\},e} \Rightarrow_S \Braket{\{S_1\},e}}
                      {,b \Rightarrow_B \top}
\\\\
\intertext{In the case that the condition $b$ is false, the rule can be rewritten to $S_2$.}
&\inference[$\text{IF-ELSE}_\bot$]{}
                      {\Braket{if(b)\{S_1\}else\{S_2\},e} \Rightarrow_S \Braket{\{S_2\},e}}
                      {,b \Rightarrow_B \bot}
\end{align*}

\subsubsection{Type Rules}

\begin{align*}
\intertext{The conditional body of a if-statement must be of type bool. The body can be of any type defined in $\Tt$. The type of the if-statement is well-typed.}
&\inference[$\text{IF}$]{\Tenv b : \Tbool &
                  \Tenv e : \Tt}
                 {\Tenv \mathbin{\text{if}} \; (b) \; \{e\}: ok}
%%%%%%%%%%%%%%%%%%%%%%%%%%%%%%%%%%%%%%%%%%%%%%%%%%%%%%%%%%%%%%%%%%%%%%%%%%%%%%%%%%%%%%%%
\intertext{The conditional body of a if-else-statement must be of type bool. The two bodies can be of any type defined in $\Tt$. The type of the if-else-statement is well-typed.}
&\inference[$\text{IF-ELSE}$]{\Tenv b : \Tbool &
                  \Tenv e_1 : \Tt &
                  \Tenv e_2 : \Tt}
                 {\Tenv \mathbin{\text{if}} \; (b) \; \{e_1\} \mathbin{\text{else}} \{e_2\}: ok}
\end{align*}

\subsection{Match Statements}
\label{subsec:matchStatements}

Match statements can be seen as syntactical sugar for multiple chained if-statements. The syntax is as follows.

\begin{verbatim}
  match <whatToMatchOn> {
    <case1> -> <actionOnCase1>
    <case2> -> <actionOnCase2>
    ...
    <caseN> -> <actionOnCaseN>
  }
\end{verbatim}

A concrete example:

\begin{verbatim}
  match (1, 2) {
    (0, n) -> // this case will never be reached
    (1, n) -> print("Case reached!");
    _ -> // default case that matches everything
  }
\end{verbatim}

\subsubsection{Delimitation}

Due to other work being deemed more importantly, match-statements will not be further developed. A future improvement to TLDR could possibly include match statements.
