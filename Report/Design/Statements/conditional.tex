\subsection{Conditional}
\subsubsection{If statements}
\label{subsec:ifStatements}

\begin{align*}
\intertext{The conditional body of a if-statement must be of type bool. The body can be of any type defined in $\Tt$. The type of the if-statement is well-typed.}
&\inference[$\text{IF}$]{\Tenv b : \Tbool &
                  \Tenv e : \Tt}
                 {\Tenv \mathbin{\text{if}} \; (b) \; \{e\}: ok}
%%%%%%%%%%%%%%%%%%%%%%%%%%%%%%%%%%%%%%%%%%%%%%%%%%%%%%%%%%%%%%%%%%%%%%%%%%%%%%%%%%%%%%%%
\intertext{The conditional body of a if-else-statement must be of type bool. The two bodies can be of any type defined in $\Tt$. The type of the if-else-statement is well-typed.}
&\inference[$\text{IF-ELSE}$]{\Tenv b : \Tbool &
                  \Tenv e_1 : \Tt &
                  \Tenv e_2 : \Tt}
                 {\Tenv \mathbin{\text{if}} \; (b) \; \{e_1\} \mathbin{\text{else}} \{e_2\}: ok}
\end{align*}

If-statements can be written either as a if-then-else statement or just as an if-statement. The following shows either way.

\begin{verbatim}
  // if-then-else statement
  if (<condition>) {<what to do if condition is true>}
  else {<what to do if condition is false>}

  // if-statement
  if (<condition>) {<what to do if condition is true>}
\end{verbatim}

A concrete example:

\begin{verbatim}
  // if-then-else statement
  if (2 + 2 = 4) {"math works!"}
  else {"something is wrong here!"}

  // if-statement
  if (remainingTime < 10) {initiateCountdown()}
\end{verbatim}

Below follows the semantics for the If-statements:

\begin{align*}
&\inference[$\text{IF}_\top$]{}
                      {\Braket{if(b)\{S\},e} \Rightarrow_S \Braket{\{S\},e}}
                      {,b \Rightarrow_B \top}
\\\\
&\inference[$\text{IF}_\bot$]{}
                      {\Braket{if(b)\{S\},e} \Rightarrow_S e}
                      {,b \Rightarrow_B \bot}
\\\\
&\inference[$\text{IF-ELSE}_\top$]{}
                      {\Braket{if(b)\{S_1\}else\{S_2\},e} \Rightarrow_S \Braket{\{S_1\},e}}
                      {,b \Rightarrow_B \top}
\\\\
&\inference[$\text{IF-ELSE}_\bot$]{}
                      {\Braket{if(b)\{S_1\}else\{S_2\},e} \Rightarrow_S \Braket{\{S_2\},e}}
                      {,b \Rightarrow_B \bot}
\end{align*}

\subsubsection{Match statements}
\label{subsec:matchStatements}

Match statements can be seen as syntactical sugar for multiple chained if-statements. The syntax is as follows.

\begin{verbatim}
  match <whatToMatchOn> {
    <case1> -> <actionOnCase1>
    <case2> -> <actionOnCase2>
    ...
    <caseN> -> <actionOnCaseN>
  }
\end{verbatim}

A concrete example:

\begin{verbatim}
  match (1, 2) {
    (0, n) -> // this case will never be reached
    (1, n) -> print("Case reached!");
    _ -> // default case that matches everything
  }
\end{verbatim}

\paragraph{Delimitation}

Due to other work being deemed more importantly, match statements will not be further developed. A future improvement to TLDR could possibly include match statements.