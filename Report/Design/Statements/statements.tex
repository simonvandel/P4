\section{Statements}\label{sec:statements}
Statements are in TLDR defined as constructs that have the possibility to change the state of the program. The state of the program is defined as the symbol by which values are associated. We call this the environment and it is formally described as:
\begin{center}
$e = \text{Symbols} \rightharpoonup \text{Stm} \times \text{Symbols}$
\end{center}
This is a partial function that maps symbols to statements and other statements. These statements are the statements that are run when the symbol is invoked. If these maps to an expression the invocation takes the value of that invocation in the current environment. This can be any expression and as explained the symbol evaluates to whatever the expression maps to. Note that a block is also an expression, see \cref{subsubsec:invocation}. If the statements evaluates only to a statement the symbol is of type \enquote{unit} and it has no value. Note that the statements can still update the state of the program during the invocation. 

If the symbol takes input arguments these are placed inside the \enquote{Symbols} on the right hand side of the arrow. In an invocation these symbols will be mapped to the values given as input parameters.

\subsection{Initialisations}\label{subsec:initialisations}
A new symbol in TLDR can be created via an initialisation. A symbol always maps to at least one statement, i.e. a symbol can not map to nothing. This means that the environment can never map a symbol to the empty set $\epsilon$. For that reason the initialisation always include an assignment with statements and/or expressions. Since symbols only can map to statements a symbol can only have a value when evaluated.

In traditional mathematical notation, the \enquote{=} symbol is also sometimes used to let certain symbols represent a more complex meaning, in order to simplify something, such as an equation or a function. When used like this, often mathematicians put the word \enquote{let} in front of a statement to denote that it is a definition. We wanted to follow this construct as well letting immutable assignment be denoted in this fashion, since they are comparable to definitions. 

But since we wanted assignments, be it mutable or immutable, to have a similarities, we chose \enquote{:=} for all assignments. This concept is less known in traditional mathematics, but is widely used in computational science. In the historically significant languages Fortran and C, the \enquote{=} symbol, was used for this. However, since we wish to keep that symbol closer to its original meaning, we needed something else. \enquote{:=} was chosen, since it is a known symbol from other languages. The asymmetry of the \enquote{:=} symbol also illustrates that it matters which side of the symbol a variable is on, as opposed to the \enquote{=} symbol.

When assigning statements we differentiates between functions and constants/variables. Whether the symbol is constant or not and whether or not it is a function is defined at initialisation. Whether it is a constant or variable binding is denoted with the keywords \enquote{let}, a constant binding, and \enquote{var}, a variable binding. If the statements are bound constant they can never be changed. This is useful, especially in mathematics where many things both functions and constants never changes. But for things like results, state and generic behaviour a variable binding is useful.

Whether a symbol is a function or not is denoted via parentheses, parentheses meaning that it is a function and no parentheses meaning that it is not. Note that a symbol can have empty parentheses and still be a function meaning that there is a difference between \enquote{let a():int := \dots} and \enquote{let a:int := \dots}. The formal syntax is as follows: 

\begin{grammar}
<Initialisation> ::= ('let' | 'var') (<FuncDecl> | <SymDecl>) ':=' <Expression>

<FuncDecl> -> <Identifier> '(' <Ids>? ')' ':' <Types>

<SymDecl> -> <Identifier> ':' <Types>
\end{grammar}
If a symbol is a constant or variable the right hand side of the \enquote{:=} is evaluated and the symbol is assigned the simplest statement that will always evaluate to the value of the evaluation. If the right hand side is evaluated to a number it is formally described as follows:
\begin{align*}
&\inference[$\text{INIT}_{SYM-A1}$]{\Braket{x,env_s} \Rightarrow_A \Braket{x',env_s}}
                         {\Braket{\Tlet \Ta := x,env_s} \Rightarrow_S \Braket{\Tlet \Ta := x',env_s}}
\\\\
&\inference[$\text{INIT}_{SYM-A2}$]{\Braket{x,env_s} \Rightarrow_A v}
                         {\Braket{\Tlet \Tx := x,env_s} \Rightarrow_S env_s'}
\\
&{\Twhere env_s' = env_{s}[\Tx \mapsto \Braket{\{n\},\epsilon}], \mathcal{N}(n) = v}
\end{align*}
The value the expression evaluates is converted via the $\mathcal{N}$ to the numeral and the symbol is assigned this as within a block. In this way an invocation of the symbol will evaluate the block and the same value each time. If the right hand side is evaluated to a boolean the formal semantics are described as follows:
\begin{align*}
&\inference[$\text{INIT}_{SYM-BOOL}$]{\Braket{b,env_s} \Rightarrow_B \Braket{b',env_s}}
                         {\Braket{\Tlet \Ta := b,env_s} \Rightarrow_S \Braket{\Tlet \Ta := b',env_s}}
\\\\
&\inference[$\text{INIT}_{SYM-\top}$]{\Braket{b,env_s} \Rightarrow_B \top}
                         {\Braket{\Tlet \Tx := x,env_s} \Rightarrow_S env_s'}
\\
&{\Twhere env_s' = env_{s}[\Tx \mapsto \Braket{\{\Ttrue\},\epsilon}]}
\\\\
&\inference[$\text{INIT}_{SYM-\bot}$]{\Braket{b,env_s} \Rightarrow_B \bot}
                         {\Braket{\Tlet \Tx := x,env_s} \Rightarrow_S env_s'}
\\
&{\Twhere env_s' = env_{s}[\Tx \mapsto \Braket{\{\Tfalse\},\epsilon}]}
\end{align*}
Again the resulting value is placed as the statement \enquote{true} or \enquote{false} inside a block, thereby always evaluating to this value.

For the representation of functions in our language, we wanted to stay as close as possible to the mathematical functions, such as \enquote{f(x,y) = z}. In order to achieve this, the type of the function arguments had to either be inherent or declared elsewhere. Since we chose to delimit ourselves from type inherence it must be declared. This is done in the same style as any type declaration, with a kolon. Functions in TLDR are treated like values, being reassignable and potentially having a function take another function as a parameter or give it as a return value.

If a symbol is initialised as a function it is dynamically scoped. This means that the statements, a symbol maps to, is evaluated when it is invoked, in the environment in which it is invoked, meaning that if the function is impure, see \cref{subsubsec:invocation}, it will use the statements the symbols maps to at the time of invocation not initialisation. This means that a function when not invoked maps to the statements it was assigned and other functions can be assigned these or use them as inputs.

\begin{align*}
&\inference[$\text{INIT}_{FUNC1}$]{}
                         {\Braket{\Tlet \Tx() := \{S\},sEnv} \Rightarrow_S sEnv'}
												 {, sEnv' = sEnv[\Tx \mapsto \Braket{S,\epsilon}]}
\end{align*}
If the symbol is a function and takes input parameters these are placed as formal parameters in the symbols ordered set the symbol maps to in the order they appear.
\begin{align*}
&\inference[$\text{INIT}_{FUNC2}$]{}
                         {\Braket{\Tlet \Tx(x) := \{S\},sEnv} \Rightarrow_S sEnv'}
												 {, sEnv' = sEnv[\Tx \mapsto \Braket{S,y}]}
\end{align*}
Expanding the amount of input parameters is a trivial task. All symbols are placed in the set in the order they appeal in the code. 
\begin{align*}
&\inference[$\text{INIT}_{FUNC3}$]{}
                         {\Braket{\Tlet \Tx(x,y) := \{S\};,sEnv} \Rightarrow_S sEnv'}
												 {, sEnv' = sEnv[\Tx \mapsto \Braket{S,\{x,y\}}]}
\end{align*}
Note that all rules are defined as constant bindings (\enquote{let}) but the exact same semantic goes for all variable bindings, since the allowance of reassignment is checked by the type system.

\begin{grammar}
<Identifier> ::= 'me'
 \alt <Id>
 \alt <Id> <Accessor>

<Id> ::= [a-zA-Z][a-zA-Z\_0-9]*-('let' | 'var' | 'int' | 'real' | 'char' | 'bool' | 'struct' | 'actor' | 'receive' | 'send' | 'spawn' | 'return' | 'for' | 'in' | 'if' | 'else' | 'while' | 'die' | 'me ')

<Ids> ::= <Identifier> (',' <Identifier>)*

<Types> ::= <Type> ('->' <Type>)*

<Type> ::= <PrimitiveType>
 \alt <Identifier>
 \alt <ListType>
 \alt <TupleType>

<TupleType> ::= '(' <Types> ')'

<ListType> ::= '\{' <Types> '\}'

<PrimitiveType> ::= 'int' | 'real' | 'char' | 'bool'
\end{grammar}

As can be seen, values can either be an immutable constant, or a muteable variable.

A constant binding can never have its value changed. For example, if \enquote{a} is bound to \enquote{5}, \enquote{a} can never refer to another value than \enquote{5} in the same lexical scope. 

The syntax for constant value assignment is as follows.
\begin{verbatim}
  let <symbolName> : <type> := <value>
\end{verbatim}
A concrete example:

\begin{verbatim}
  let x : int := 2
\end{verbatim}

A variable binding can always change the value it refers to. For example, if \enquote{b} is bound to \enquote{2}, it is perfectly possible to later in the source code refer to {10}.

The syntax for variable value assignment is as follows.

\begin{verbatim}
  var <symbolName> : <type> := <value>

// a later reassignment
  <symbolName> := <value>
\end{verbatim}
A concrete example:

\begin{verbatim}
  var a : int := 2
  
// a later reassignment
  a := 2
\end{verbatim}

\subsection{Assignment}\label{subsec:assignment}
\begin{align*}
&\inference[ASS]{\Braket{S,e} \Rightarrow_S \Braket{S',e'}}
                 {\Braket{\Tx := S,e} \Rightarrow_S \Braket{\Tx := S',e'}}
\\\\
&\inference[ASS]{\Braket{S,e} \Rightarrow_S v}
                 {\Braket{\Tx := S,e} \Rightarrow_S e'}
								 {, e' = e[\Tx \mapsto \Braket{\{n\},\epsilon}], \mathcal{N}(n) = v}
\end{align*}


\begin{align*}
\intertext{}
&\inference[LET]{}
                 {\Tenv \mathbin{\text{let}} \; e:T: ok}
\\\\
&\inference[VAR]{}
                 {\Tenv \mathbin{\text{var}} \; e:T: ok}
\\\\
&\inference[LET1]{\Tenv e_1: \Tt & \Tenv e_2: \Tt}
                 {\Tenv \mathbin{\text{let}} \; e_1 := e_2: ok}
\\\\
&\inference[VAR1]{\Tenv e_1: \Tt & \Tenv e_2: \Tt}
                 {\Tenv \mathbin{\text{var}} \; e_1 := e_2: ok}
\end{align*}

\subsubsection{Functions}
\label{subsec:functions}

Functions can be declared in two ways. By separating the type signature and the function body or by combining the signature and the function body.

\paragraph{Separated function declaration}

The syntax for the separated function declaration is as follows. The type signature and the body must be declared in the same lexical scope.

\begin{verbatim}
  <funcName> : <typeSignature>;
  let <funcName>(<parameterList>) := {<body>};
\end{verbatim}

A concrete example:

\begin{verbatim}
  plus : int -> int -> Int;
  let plus(x, y) := {x + y};
\end{verbatim}


\paragraph{Combined function declaration}

The syntax for the combined function declaration is as follows.

\begin{verbatim}
  let <funcName>(<parameterList>) : <typeSignature> := {<body>};
\end{verbatim}

A concrete example:

\begin{verbatim}
  let plus(x, y) : Int -> Int -> Int := {x + y};
\end{verbatim}

%mainfile: ../../master.tex
\subsection{Structures}
\label{subsec:structs}

In TLDR there are three ways to do encapsulation. Actors, Tuples and structs. Structs are unique by being fully accessible within the scope, and having named fields. Structs are especially useful in TLDR for creating messages.

\subsubsection{Defining Structures}
\label{sec:defStructures}

\subsubsection{Syntax}

Structures are defined by using the \enquote{struct} keyword. The grammar for declaring structs are as follows:

\begin{grammar}
  <Struct> ::= 'struct' <Identifier> ':= \{' <TypeDecls> '\}'
\end{grammar}

And a concrete example:

\begin{lstlisting}[style=TLDR]
  struct Person := {Name:[char]; Age:int}
\end{lstlisting}

\subsubsection{Semantics}

having these semantics:

\begin{align*}
\intertext{In the case that we have multiple S of assignments, we can rewrite the T to now map to new s' that includes x, in the new st' environment and the rest of declaration statements}
&\inference[$\text{STRUCT}$]{}
                            {\Braket{\Tstruct T := \{x:T';S\}, env_s, st} \Rightarrow_S \Braket{\Tstruct T := \{S\},env_s,st'}}
\\
&{\Twhere st' = st[T \mapsto s'],st(T) = s,s' = s \cup x]}
\\\\
\intertext{In the case that we have multiple S of assignments, we can rewrite the T to now map to new s' that includes x, in the new st' environment}
&\inference[$\text{STRUCT}$]{}
                            {\Braket{\Tstruct T := \{x:T\}, env_s, st} \Rightarrow_S \Braket{env_s,st'}}
\\
&{\Twhere st' = st[T \mapsto s'],st(T) = s,s' = s \cup x]}
\end{align*}

\begin{align*}
&\inference[$\text{STRUCT}$]{}
                            {\Braket{\Tstruct T_1 := (x:T_2;S), env_s} \Rightarrow_S \Braket{\Tstruct T_1 := (S),env_s'}}
\\
&{\Twhere env_s' = env_s[T \mapsto \Braket{\epsilon,s'}],env_s(T) = s,s' = s \cup x}
\\\\
&\inference[$\text{STRUCT}$]{}
                            {\Braket{\Tlet \; \Ta:T := (S),env_s} \Rightarrow_S \Braket{\Ta := (S),env_s'}}
\\
&{env_s' = env_s[x \mapsto (S,s)],env_s(T) = (\epsilon,s)]}
\\\\
&\inference[$\text{STRUCT}$]{\Braket{x,sEnv} \Rightarrow_a \Braket{x',sEnv}}
                            {\Braket{\Tlet \Ta := (f := x;S),env_s} \Rightarrow_S \Braket{\Tlet \Ta := (f := x';S),env_s}}
\\\\
&\inference[$\text{STRUCT}$]{\Braket{x,sEnv} \Rightarrow_a n}
                            {\Braket{\Tlet \Ta := (f := x;S),env_s} \Rightarrow_S \Braket{\Tlet \Ta := (S),env_s[s.f \mapsto {n}]}}
\\\\
&\inference[$\text{STRUCT}$]{sEnv(x) = \Braket{(S),s}}
                            {\Braket{\Tlet \Ta := (f := x;S),env_s} \Rightarrow_S \Braket{\Tlet \Ta.f.s := s;\Ta := (S),env_s}}
\\\\
&\inference[$\text{STRUCT}$]{\Braket{x,sEnv} \Rightarrow_a n}
                            {\Braket{\Tlet \Ta := (f := x;S),env_s} \Rightarrow_S \Braket{\Tlet s := (S),env_s[s.f \mapsto {n}]}}
\\\\
&\inference[$\text{LIST}$]{}
                            {\Braket{s:[T] := [x_n,..,x_m],sEnv_1} \Rightarrow_S \Braket{s:[T] := [x_{n+1},..,x_m],sEnv_1}}
\\
&{sEnv_1(s) = \Braket{\epsilon,\epsilon,sEnv_2},sEnv_2' = sEnv_2[n \mapsto x_n],sEnv_1'(s) = \Braket{\epsilon,\epsilon,sEnv_2'}}
\\\\
&\inference[$\text{TUPLE}$]{}
                           {\Braket{s:[T] := [x_n,..,x_m],sEnv_1} \Rightarrow_S \Braket{s:[T] := [x_{n+1},..,x_m],sEnv_1}}
\\
&{sEnv_1(s) = \Braket{\epsilon,\epsilon,sEnv_2},sEnv_2' = sEnv_2[n \mapsto x_n],sEnv_1'(s) = \Braket{\epsilon,\epsilon,sEnv_2'}}
\end{align*}

\subsubsection{Type Rules}

\begin{align*}
\intertext{The elements can be of any type defined in $\Tt$}
&\inference[STRUCT]{E[s \mapsto (e_1:\Tt_1;e_2:\Tt_2;...;e_n:\Tt_n) \rightarrow ok]\vdash S : ok & }
                 {\Tenv \mathbin{\text{struct s}} := \{e_1:\Tt_1;e_2:\Tt_2;...;e_n:\Tt_n\}; S: ok}
\end{align*}



\subsubsection{Initialising Structures}
\label{sec:initStructures}

Structs can be initialised and assigned to symbols using either a constant assignment or a variable assignment. Structs initialised as a constant assignment cannot change any of the fields of the structs; the struct is immutable. Structs initialised as a variable assignment can change all of its fields at any time; the struct is mutable.

\subsubsection{Syntax}

The syntax for initialising a struct is as follow.

\begin{grammar}
<StructLiteral> ::= '(' (<Reassignment>';')* ')' (':' <Identifier>)?
\end{grammar}


With concrete examples for immutables:

\begin{verbatim}
  let Alice:Person := (Name := "Alice"; Age := 20);
\end{verbatim}

And for mutable struct is as follow.

\begin{verbatim}
  var Alice:Person := (Name := "Alice"; Age := 20);
\end{verbatim}

And for usage in lists.

\begin{verbatim}
  [(Name := "Alice"; Age := 20):Person];
\end{verbatim}

\subsubsection{Access to Structure Fields}
\label{sec:accessStructFields}

Fields can be access using the following syntax.

\begin{verbatim}
  Alice.Name; // "Alice"
\end{verbatim}

Structs declared as mutable can have fields reassigned using the following syntax.

\begin{verbatim}
  Alice.Name; // "Alice"
  Alice.Name := "Bob";
  Alice.Name; // "Bob";
\end{verbatim}
 
\subsubsection{Semantics}

\begin{align*}
\intertext{In the case that we have multiple S of assignments, we can rewrite the s to now having an accessor that maps to value of the first assignment and the rest of pending assignments in s}
&\inference[$\text{STRUCT}$]{}
                            {\Braket{\Tlet \; s:T := (f := x;S),env_s,st} \Rightarrow_S \Braket{s.f := x;\Tlet \; s:T := (S),env_s,st}}
\\\\
\intertext{In the case that we have only one assignment, we can rewrite the s to now having an accessor that maps to value of the assignment}
&\inference[$\text{STRUCT}$]{}
                            {\Braket{\Tlet \; s:T := (f := x),env_s,st} \Rightarrow_S \Braket{s.f := x,env_s,st}}
%\\\\
%&\inference[$\text{STRUCT}$]{}
%                            {\Braket{(f := x;S),env_s,st} \Rightarrow_S \Braket{s.f := x;\Tlet \; s:T := (S),env_s,st}}
%\\\\
%&\inference[$\text{STRUCT}$]{}
%                            {\Braket{(f := x):T,env_s,st} \Rightarrow_S \Braket{s.f := x,env_s,st}}
\end{align*}

\subsubsection{Type Rules}

Each e of type t is matching the declared struct types, evaluates to the type of the declared struct.

\begin{align*}
&\inference[STRUCTLITERAL]{\Tenv (e_1 : \Tt_1;e_2 : \Tt_2;...;e_n : \Tt_n):\Tt'}
                 {\Tenv \mathbin{\text{(}} e_1; e_2;...;e_n\mathbin{\text{)}}:\Tt': \Tt'}
\end{align*}

\subsubsection{Comparison of Structs}
\begin{align*}
&\inference[STRUCT]{\Tenv e_1: \Tt & \Tenv e_2: \Tt}
                 {\Tenv e_1 = e_2: \Tbool}
\end{align*}

\subsubsection{For-loop Statements}
\label{subsec:forLoopStatements}

A for-loop iterates through a list of elements.

The for loop statement has the following syntax.

\begin{verbatim}
  for <element> in <collection> {<loopBody>}
\end{verbatim}

A concrete example:

\begin{verbatim}
  for i:int in [0..10]:[int] { /* Do stuff with i elements */ }
\end{verbatim}
\subsubsection{While-loop Statements}
\label{subsec:whileLoopStatements}

The while loop statement runs a block of statements until a boolean expression returns false.

\paragraph{While loop}
\label{sec:whileLoop}

The while statement has the following syntax.

\begin{verbatim}
  while (<boolExp>) {<loopBody>}
\end{verbatim}

A concrete example:

\begin{verbatim}
  var i:int := 0;
  while (i < 10) {i := i + 1}
\end{verbatim}

\subsubsection{Semantics}

\begin{align*}
&\inference[$\text{WHILE}_\top$]{e \vdash b \Rightarrow_B \top}
                       {\Braket{\Twhile(b)\{S\},e} \Rightarrow_S \Braket{\{S\}; \Twhile (b)\{S\},e}}
\\\\
&\inference[$\text{WHILE}_\bot$]{e \vdash b \Rightarrow_B \bot}
                       {\Braket{\Twhile(b)\{S\},e} \Rightarrow_S e}
\end{align*}
%mainfile: ../master.tex
\subsection{If-statements}
\label{subsec:ifStatements}

If-statements can be written either as a if-else statement or just as an if-statement. In an if-else statement, the body just after the condition is evaluated if the condition evaluates to the boolean value \emph{true}. If the condition evaluates to \emph{false}, the body after the else keyword is evaluated. In an if-statement, the body after the condition is run, if the condition has the value \emph{true}. If the condition has value \emph{false}, nothing is evaluated.

\subsubsection{Syntax}

\begin{grammar}
<If> ::= 'if' <Expression> <Block>

<IfElse> ::= 'if' <Expression> <Block> 'else' <Block>
\end{grammar}

A concrete example:

\begin{verbatim}
  // if-then-else statement
  if (2 + 2 = 4) {"math works!"}
  else {"something is wrong here!"}

  // if-statement
  if (remainingTime < 10) {initiateCountdown()}
\end{verbatim}

\subsubsection{Semantics}

\begin{align*}
\intertext{In the case that the condition $b$ is true, the rule for the if-statement can simply be rewritten to the statement $S$.}
&\inference[$\text{IF}_\top$]{}
                      {\Braket{if(b)\{S\},sEnv} \Rightarrow_S \Braket{\{S\},sEnv}}
                      {,b \Rightarrow_B \top}
\\\\
\intertext{In the case that the condition $b$ is false, nothing is evaluated, and the if-statement is rewritten to the symbol environment $sEnv$.}
&\inference[$\text{IF}_\bot$]{}
                      {\Braket{if(b)\{S\},sEnv} \Rightarrow_S sEnv}
                      {,b \Rightarrow_B \bot}
\\\\
\intertext{In the case that the condition $b$ is true, the rule for the if-else statement is rewritten to $S_1$.}
&\inference[$\text{IF-ELSE}_\top$]{}
                      {\Braket{if(b)\{S_1\}else\{S_2\},e} \Rightarrow_S \Braket{\{S_1\},e}}
                      {,b \Rightarrow_B \top}
\\\\
\intertext{In the case that the condition $b$ is false, the rule can be rewritten to $S_2$.}
&\inference[$\text{IF-ELSE}_\bot$]{}
                      {\Braket{if(b)\{S_1\}else\{S_2\},e} \Rightarrow_S \Braket{\{S_2\},e}}
                      {,b \Rightarrow_B \bot}
\end{align*}

\subsubsection{Type Rules}

\begin{align*}
\intertext{The conditional body of a if-statement must be of type bool. The body can be of any type defined in $\Tt$. The type of the if-statement is well-typed.}
&\inference[$\text{IF}$]{\Tenv b : \Tbool &
                  \Tenv e : \Tt}
                 {\Tenv \mathbin{\text{if}} \; (b) \; \{e\}: ok}
%%%%%%%%%%%%%%%%%%%%%%%%%%%%%%%%%%%%%%%%%%%%%%%%%%%%%%%%%%%%%%%%%%%%%%%%%%%%%%%%%%%%%%%%
\intertext{The conditional body of a if-else-statement must be of type bool. The two bodies can be of any type defined in $\Tt$. The type of the if-else-statement is well-typed.}
&\inference[$\text{IF-ELSE}$]{\Tenv b : \Tbool &
                  \Tenv e_1 : \Tt &
                  \Tenv e_2 : \Tt}
                 {\Tenv \mathbin{\text{if}} \; (b) \; \{e_1\} \mathbin{\text{else}} \{e_2\}: ok}
\end{align*}

\subsection{Match Statements}
\label{subsec:matchStatements}

Match statements can be seen as syntactical sugar for multiple chained if-statements. The syntax is as follows.

\begin{verbatim}
  match <whatToMatchOn> {
    <case1> -> <actionOnCase1>
    <case2> -> <actionOnCase2>
    ...
    <caseN> -> <actionOnCaseN>
  }
\end{verbatim}

A concrete example:

\begin{verbatim}
  match (1, 2) {
    (0, n) -> // this case will never be reached
    (1, n) -> print("Case reached!");
    _ -> // default case that matches everything
  }
\end{verbatim}

\subsubsection{Delimitation}

Due to other work being deemed more importantly, match-statements will not be further developed. A future improvement to TLDR could possibly include match statements.
