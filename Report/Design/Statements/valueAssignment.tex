\subsection{Initialisations}\label{subsec:initialisations}
A new symbol in TLDR can be created via an initialisation. A symbol always maps to at least one statement, i.e. a symbol cannot map to nothing. This means that the environment can never map a symbol to the empty set $\epsilon$. For that reason, the initialisation always includes an assignment with statements and/or expressions. Since symbols can only map statements, a symbol can only have a value when evaluated.

In traditional mathematical notation, the \enquote{=} symbol is also sometimes used to let certain symbols represent a more complex meaning, in order to simplify something, such as an equation or a function. When used like this, often mathematicians put the word \enquote{let} in front of a statement to denote that it is a definition. We wanted to follow this construct as well letting immutable assignment be denoted in this fashion, since they are comparable to definitions. 

But, since we wanted assignments, be it mutable or immutable, to have similarities, we chose \enquote{:=} for all assignments. This concept is less known in traditional mathematics, but is widely used in computational science. In the historically significant languages such as Fortran and C, the \enquote{=} symbol, was used for this. However, since we wish to keep that symbol closer to its original meaning, we needed something else. \enquote{:=} was chosen, since it is a known symbol from other languages. The asymmetry of the \enquote{:=} symbol also illustrates that it matters which side of the symbol a variable is on, as opposed to the \enquote{=} symbol.

When assigning statements, we differentiate between functions and constants/variables. Whether it is a constant or variable binding it is denoted with the keywords \enquote{let} for a constant binding and \enquote{var} for a variable binding. If the statements are bound as a constant they can never be changed. This is useful, especially in mathematics where many things both functions and constants never changes. But for things like results, state and generic behaviour a variable binding is useful.

Whether a symbol is a function or not is denoted via parentheses, parentheses meaning that it is a function and no parentheses meaning that it is not. Note that a symbol can have empty parentheses and still be a function, meaning that there is a difference between \enquote{let a():int := \dots} and \enquote{let a:int := \dots}.

\begin{grammar}
<Initialisation> ::= ('let' | 'var') (<FuncDecl> | <SymDecl>) ':=' <Expression>

<FuncDecl> -> <Identifier> '(' <Ids>? ')' ':' <Types>

<SymDecl> -> <Identifier> ':' <Types>
\end{grammar}

If a symbol is a constant or variable, the right hand side of the \enquote{:=} is evaluated and the symbol is assigned the simplest statement that will always evaluate to the value of the evaluation. If the right hand side is evaluated to a number, it is formally described as follows:

\begin{align*}
&\inference[$\text{INIT}_{SYM-A1}$]{\Braket{x,sEnv} \Rightarrow_A \Braket{x',sEnv}}
                         {\Braket{\Tlet \Ta := x,sEnv} \Rightarrow_S \Braket{\Tlet \Ta := x',sEnv}}
\\\\
&\inference[$\text{INIT}_{SYM-A2}$]{\Braket{x,sEnv} \Rightarrow_A v}
                         {\Braket{\Tlet \Ta := x,sEnv} \Rightarrow_S sEnv'}
\\
&{\Twhere sEnv' = env_{s}[\Ta \mapsto \Braket{\{n\},\epsilon}], \mathcal{N}(n) = v}
\end{align*}

The value the expression evaluates to is converted, via the $\mathcal{N}$ function, to the numeral and the symbol is assigned this as within a block. In this way an invocation of the symbol will evaluate the block and the same value each time. If the right hand side is evaluated to a boolean the formal semantics are described as follows:

\begin{align*}
&\inference[$\text{INIT}_{SYM-BOOL}$]{\Braket{b,sEnv} \Rightarrow_B \Braket{b',sEnv}}
                         {\Braket{\Tlet \Ta := b,sEnv} \Rightarrow_S \Braket{\Tlet \Ta := b',sEnv}}
\\\\
&\inference[$\text{INIT}_{SYM-\top}$]{\Braket{b,sEnv} \Rightarrow_B \top}
                         {\Braket{\Tlet \Ta := b,sEnv} \Rightarrow_S sEnv'}
\\
&{\Twhere sEnv' = sEnv[\Ta \mapsto \Braket{\{\Ttrue\},\epsilon}]}
\\\\
&\inference[$\text{INIT}_{SYM-\bot}$]{\Braket{b,sEnv} \Rightarrow_B \bot}
                         {\Braket{\Tlet \Ta := b,sEnv} \Rightarrow_S sEnv'}
\\
&{\Twhere sEnv' = sEnv[\Ta \mapsto \Braket{\{\Tfalse\},\epsilon}]}
\end{align*}

Again, the resulting value is placed as the statement \enquote{true} or \enquote{false} inside a block, thereby always evaluating to this value.

For the representation of functions in our language, we wanted to stay as close as possible to the mathematical functions, such as \enquote{f(x,y) = z}. In order to achieve this, the type of the function arguments had to either be implicit or declared elsewhere. Since we chose to delimit ourselves from type inherence it must be declared. This is done in the same style as any type declaration, with a colon. Functions in TLDR are treated like values, being reassignable and potentially having a function take another function as a parameter or give it as a return value.

If a symbol is initialised as a function it is dynamically scoped. This means that the statements, that a symbol maps to, are evaluated when it is invoked, in the environment in which it is invoked, meaning that if the function is impure, see \cref{subsubsec:invocation}, it will use the statements, that the symbols map to at the time of invocation, not at the time of initialisation. This means that a function when not invoked maps to the statements it was assigned and other functions can be assigned these or use them as inputs.

\begin{align*}
&\inference[$\text{INIT}_{FUNC1}$]{}
                         {\Braket{\Tlet \Tx() := \{S\},sEnv} \Rightarrow_S sEnv'}
												 {, sEnv' = sEnv[\Tx \mapsto \Braket{S,\epsilon}]}
\end{align*}

If the symbol is a function and takes input parameters these are placed as formal parameters in the symbols ordered set the symbol maps to in the order they appear.

\begin{align*}
&\inference[$\text{INIT}_{FUNC2}$]{}
                         {\Braket{\Tlet \Tx(x) := \{S\},sEnv} \Rightarrow_S sEnv'}
												 {, sEnv' = sEnv[\Tx \mapsto \Braket{S,y}]}
\end{align*}
Expanding the amount of input parameters is a trivial task. All symbols are placed in the set in the order they appear in the code. 

\begin{align*}
&\inference[$\text{INIT}_{FUNC3}$]{}
                         {\Braket{\Tlet \Tx(x,y) := \{S\};,sEnv} \Rightarrow_S sEnv'}
												 {, sEnv' = sEnv[\Tx \mapsto \Braket{S,\{x,y\}}]}
\end{align*}

There was an initial push towards allowing signature declarations to occur completely seperated from the function declarations, in order to make it even further resemble mathematical notation. This was not included, since the long term goal is type inherency, by which this signature would become obsolete. Furthermore, the arguement of moving the signature away to resemble mathematics, becomes invalid since any move that actually achieves this, would greatly decrease readability, and if the signature is kept close to the function, it might as well have been tied directly to the function. In the case of the function having an intuitive signature, the inclusion would not decrease readability by any significance, since it is not intertwined with the function declaration.\\

Note that all rules are defined as constant bindings (\enquote{let}) but the exact same semantic holds for all variable bindings, since the allowance of reassignment is checked by the type system.

\begin{grammar}
<Identifier> ::= 'me'
 \alt <Id>
 \alt <Id> <Accessor>

<Id> ::= [a-zA-Z][a-zA-Z\_0-9]*-('let' | 'var' | 'bool' | 'integer' | 'real' | 'char' | 'struct' | 'actor' | 'receive' | 'send' | 'spawn' | 'return' | 'for' | 'in' | 'if' | 'else' | 'while' | 'die' | 'me');

<Ids> ::= <Identifier> (',' <Identifier>)*

<Types> ::= <Type> ('->' <Type>)*

<Type> ::= <Primitive>
 \alt <Identifier>
 \alt <ListType>
 \alt <TupleType>

<TupleType> ::= '(' <Types> ')'

<ListType> ::= '[' <Types> ']'

<Primitive> ::= 'int' | 'real' | 'char' | 'bool'
\end{grammar}

As can be seen, values can either be an immutable constant, or a mutable variable.

A constant binding can never have its value changed. For example, if \enquote{a} is bound to \enquote{5}, \enquote{a} can never refer to another value than \enquote{5} in the same lexical scope. 

The syntax for constant value assignment is as follows.
\begin{verbatim}
  let <symbolName> : <type> := <value>
\end{verbatim}
A concrete example:

\begin{verbatim}
  let x : int := 2
\end{verbatim}

A variable binding can always change the value it refers to. For example, if \enquote{b} is bound to \enquote{2}, it is perfectly possible to later in the source code refer to {10}.

The syntax for variable value assignment is as follows.

\begin{verbatim}
  var <symbolName> : <type> := <value>

// a later reassignment
  <symbolName> := <value>
\end{verbatim}
A concrete example:

\begin{verbatim}
  var a : int := 2
  
// a later reassignment
  a := 2
\end{verbatim}

\subsection{Assignment}\label{subsec:assignment}
As mentioned in \cref{subsec:initialisations}, variables can be assigned a value after their initialisation. This is only the case if the right hand side of the assignment is a value and of the same type as the variable.
This is the grammar, specifying the syntax for assignment:
\begin{grammar}
<Reassignment> ::= <Identifier>  ':=' <Expression>
\end{grammar}
Assignment uses the \enquote{:=} operator. The left hand side of the operator is the symbol to be assigned the value. The right hand side is an expression which is the value to be assigned.

Since space is not allocated in semantics but the functions merely maps symbols to values the semantic rules for assignment are practically the same as initialisations:
\begin{align*}
&\inference[$\text{ASS}_{SYM-A1}$]{\Braket{x,sEnv} \Rightarrow_A \Braket{x',sEnv}}
                         {\Braket{\Ta := x,sEnv} \Rightarrow_S \Braket{\Ta := x',sEnv}}
\\\\
&\inference[$\text{ASS}_{SYM-A2}$]{\Braket{x,sEnv} \Rightarrow_A v}
                         {\Braket{\Ta := x,sEnv} \Rightarrow_S sEnv'}
\\
&{\Twhere sEnv' = env_{s}[\Ta \mapsto \Braket{\{n\},\epsilon}], \mathcal{N}(n) = v}
\end{align*}

\begin{align*}
&\inference[$\text{ASS}_{SYM-BOOL}$]{\Braket{b,sEnv} \Rightarrow_B \Braket{b',sEnv}}
                         {\Braket{\Ta := b,sEnv} \Rightarrow_S \Braket{\Ta := b',sEnv}}
\\\\
&\inference[$\text{ASS}_{SYM-\top}$]{\Braket{b,sEnv} \Rightarrow_B \top}
                         {\Braket{\Ta := b,sEnv} \Rightarrow_S sEnv'}
\\
&{\Twhere sEnv' = sEnv[\Ta \mapsto \Braket{\{\Ttrue\},\epsilon}]}
\\\\
&\inference[$\text{ASS}_{SYM-\bot}$]{\Braket{b,sEnv} \Rightarrow_B \bot}
                         {\Braket{\Ta := b,sEnv} \Rightarrow_S sEnv'}
\\
&{\Twhere sEnv' = sEnv[\Ta \mapsto \Braket{\{\Tfalse\},\epsilon}]}
\end{align*}

\begin{align*}
&\inference[$\text{INIT}_{FUNC1}$]{}
                         {\Braket{\Tx() := \{S\},sEnv} \Rightarrow_S sEnv'}
												 {, sEnv' = sEnv[\Tx \mapsto \Braket{S,\epsilon}]}
\end{align*}

\begin{align*}
&\inference[$\text{INIT}_{FUNC2}$]{}
                         {\Braket{\Tx(x) := \{S\},sEnv} \Rightarrow_S sEnv'}
												 {, sEnv' = sEnv[\Tx \mapsto \Braket{S,y}]}
\end{align*} 

\begin{align*}
&\inference[$\text{INIT}_{FUNC3}$]{}
                         {\Braket{\Tx(x,y) := \{S\};,sEnv} \Rightarrow_S sEnv'}
												 {, sEnv' = sEnv[\Tx \mapsto \Braket{S,\{x,y\}}]}
\end{align*}

%\begin{align*}
%&\inference[ASS]{\Braket{S,sEnv} \Rightarrow_S \Braket{S',sEnv'}}
                 %{\Braket{\Tx := S,sEnv} \Rightarrow_S \Braket{\Tx := S',sEnv'}}
%\\\\
%&\inference[ASS]{\Braket{S,sEnv} \Rightarrow_a v}
                 %{\Braket{\Tx := S,sEnv} \Rightarrow_S sEnv'}
								 %{, e' = e[\Tx \mapsto \Braket{\{n\},\epsilon}], \mathcal{N}(n) = v}
%\end{align*}
%
%\begin{align*}
%&\inference[ASS]{\Braket{S,e} \Rightarrow_S v}
                 %{\Braket{\Tx := S,e} \Rightarrow_S e'}
								 %{, e' = e[\Tx \mapsto \Braket{\{n\},\epsilon}], \mathcal{N}(n) = v}
%\\\\
%&\inference[ASS]{\Braket{S,e} \Rightarrow_S v}
                 %{\Braket{\Tx := S,e} \Rightarrow_S e'}
								 %{, e' = e[\Tx \mapsto \Braket{\{n\},\epsilon}], \mathcal{N}(n) = v}
%\end{align*}
%
%\begin{align*}
%&\inference[$\text{ASS}_{SYM-BOOL}$]{\Braket{b,sEnv} \Rightarrow_B \Braket{b',sEnv}}
                         %{\Braket{\Tat := b,sEnv} \Rightarrow_S \Braket{\Tat := b',sEnv}}
%\\\\
%&\inference[$\text{ASS}_{SYM-\top}$]{\Braket{b,sEnv} \Rightarrow_B \top}
                         %{\Braket{\Tx := x,sEnv} \Rightarrow_S sEnv'}
%\\
%&{\Twhere sEnv' = env_{s}[\Tx \mapsto \Braket{\{\Ttrue\},\epsilon}]}
%\\\\
%&\inference[$\text{ASS}_{SYM-\bot}$]{\Braket{b,sEnv} \Rightarrow_B \bot}
                         %{\Braket{\Tx := x,sEnv} \Rightarrow_S sEnv'}
%\\
%&{\Twhere sEnv' = env_{s}[\Tx \mapsto \Braket{\{\Tfalse\},\epsilon}]}
%\end{align*}

This is the type rule for assignment:
\begin{align*}
&\inference[ASS]{\Tenv I : T & \Tenv e: T}
                 {\Tenv \text{\{I := e\}} : ok}
\\\\
\end{align*}
