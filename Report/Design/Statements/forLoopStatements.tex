%mainfile: ../master.tex
\subsection{For-loop}
\label{subsec:forLoopStatements}

A for-loop iterates through a list of elements.

The for loop statement follows this grammar:

\begin{grammar}
<ForIn> ::= 'for' <Identifier> 'in' <List | Identifier> <Block>
\end{grammar}

A concrete example:

\begin{verbatim}
  sum:int = 0;
  for i:int in [0..10]:[int] { sum = sum + i }
\end{verbatim}

Here follows the semantics for the for-loop statement:

In the case that the list contains more than one element, the rule can be rewritten as x now being the head of the list, the statements of the block and the for-loop statement again with rest of the list.

\begin{align*}
&\inference[$\text{FORIN}_{1}$]{}
                       {\Braket{\Tfor \Tx \Tin [n_1,\dots,n_m]\{S\},sEnv} \Rightarrow_S \Braket{x := n_1;\{S\}; \Tfor \mathbin{\text{x}} \mathbin{\text{in}} [n_2,\dots,n_m]\{S\},sEnv}}
\\\\
&\inference[$\text{FORIN}_{2}$]{}
                       {\Braket{\Tfor \mathbin{\text{x}}\mathbin{\text{in}} [n_m]\{S\},sEnv} \Rightarrow_S \Braket{x := n_m;\{S\},sEnv}}
\end{align*}

Initially a parallel for-loop was included in the language, and this is still potentially a desired feature. It was removed however, since the languages focuses on parallelism through actors, and so it was deemed too time consuming to focus on also parallelising statements and expressions.

\begin{align*}
\intertext{The element i of the ForIn construct must be of the same type as L. The body of the ForIn-loop can be of any type defined in $\Tt'$}
&\inference[FORIN]{\Tenv t : \Tt &
                   \Tenv L : [\Tt] &
                  \Tenv e : \Tt'}
                 {\Tenv \mathbin{\text{for}} \; (t \; \mathbin{\text{in}} L) \; {e}: ok}
\end{align*}

\subsubsection{Loop Expressions}
\begin{align*}
\intertext{For loops can only work on lists. The counter variable is the type of a element in the list being iterated.}
&\inference[FOR]{\Tenv l : [\Tpt]}
                 {\Tenv \mathbin{\text{for}} \; \mathbin{\text{x}} \; \mathbin{\text{in}} \; {l} \; : ok },	 \Tenv x : \Tpt
\end{align*}
