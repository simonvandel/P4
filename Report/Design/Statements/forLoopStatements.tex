\subsection{For-loop Statements}
\label{subsec:forLoopStatements}

A for-loop iterates through a list of elements.

The for loop statement follows this grammar:


Which can informally be viewed as:

\begin{verbatim}
  for <element> in <collection> {<loopBody>}
\end{verbatim}

A concrete example:

\begin{verbatim}
  for i:int in [0..10]:[int] { /* Do stuff with i elements */ }
\end{verbatim}

Here follows the semantics for the for-loop statement:

\begin{align*}
&\inference[$\text{FOR}_{1}$]{}
                       {\Braket{\Tfor \Tx \Tin [n_1,\dots,n_m]\{S\},e} \Rightarrow_S \Braket{x := n_1;\{S\}; \Tfor \mathbin{\text{x}} \mathbin{\text{in}} [n_2,\dots,n_m]\{S\},e}}
\\\\
&\inference[$\text{FOR}_{2}$]{}
                       {\Braket{\Tfor \mathbin{\text{x}}\mathbin{\text{in}} [n_m]\{S\},e} \Rightarrow_S \Braket{x := n_m;\{S\},e}}
\end{align*}

Initially a parallel for-loop was included in the language, and this is still potentially a wanted feature. It was removed however, since the languages focuses on parallelism through actors, and so it was deemed too time consuming to focus on also parallelising statements and expressions.