\section{Type System}

The goal of the type system is to minimise programming errors, but on the other hand not being too restrictive so the expressiveness of the language is reduced. 

The Language Described in this Report is a statically typed language. Statically typed languages have the following nice properties, that do not apply to dynamically typed languages.

\begin{itemize}
  \item Type errors are presented at the soonest possible level: compile-time. This makes it impossible to have a program stop unexpectedly because of a type error at run-time
  \item Because of the fact that types are checked at compile time, no overhead is imposed to the compiled program at run-time
\end{itemize}

Of course, dynamically typed languages have its uses. One can argue that prototyping a program is done faster in a dynamically typed language. The trade-off is usually expressiveness for safety and performance.

In order to reduce the mental overhead and the number of explicit type declarations to write in a language, type inference can be implemented in the compiler. Having almost all types inferred by the compiler makes the language have more of the same feel as a dynamically typed language, in that there is less mental overhead for the user. Using a statically typed language with type inference brings all the beneficial properties of statically typed language, combined with a \enquote{lighter} feel.

\subsection{Strictness}

The type system defined in this report can be described as a \emph{strong} type system, as opposed to a \emph{weak} type system. No formal definitions of such categorisations of type systems exist, so this report will use the following understanding of \emph{weak} and \emph{strong} type systems. A type system goes from \emph{weak} to \emph{strong} as the number of undefined behaviour, unpredictable behaviour or implicit conversions between types approach zero.

Using the above definition the type system can be categorised as \emph{strong}, as only one implicit conversion is made between values of type \emph{int} to values of type \emph{real}, whenever number precision can be preserved. Other than that, the type system should be predictable.

\subsection{Meta type}
\label{sub:meta_type}

It has been decided to design the language in a manner that insures that all type checking can be done at compile time. To insures this all identifier's must have explicit type declaration. A meta type is in this report defined as a type which can be an instance of any other type in the language.  The functionally which spawns a new actor needs to know of which type the actor should be of, in this case a meta type would be useful to describe the actual type of the actor.
So to imagine a meta type in this language, it would look something like this:

\begin{verbatim}
  let x:Type := Int
\end{verbatim}
In fact $typeX$ is just a placeholder for a type the following expression should evaluate to int and the declaration of $y$ would be fine. Bu notice that type is declared using the var keyword which makes an variable mutable. If $typeX$ is mutable it can change state doing runtime, and in this case the type of $y$ can not be determined at compile time, which has disagrees with the premise mentioned before. 

\begin{verbatim}
  var typeX:Type := Int
  let y:typeX := 4
\end{verbatim}
Different solutions for this problem have been considered, one way of doing it is to disallow mutable meta type. This solution insures that the meta type can't change at runtime, and there for the type can be determined at compile time. Another way of doing it could be to disallow a variable to be type declared using a meta type. Because of orthogonality it has been decided to not implement any kind of meta type, with this solution there will be no confusion about how a meta type can be declared and no to be declared. 
In \cref{sub:constructionOfAnActor} it's describe how it's possible to spawn an actor without the use of a meta type.


\subsection{Type Rules}

\subsubsection{Common Notation}
\textbf{Operators}
\begin{align*}
&\textrm{let} \;\; AOP = [+, -, *, /, \%, \;^\wedge{} \;]
&
&\textrm{let} \;\; BAOP = [=, !=, <, <=, >, >=]
\\\\
&\textrm{let} \;\; BBOP = [\textrm{AND, NAND, OR, NOR, XOR}]
\end{align*}
\textbf{Types}
\begin{align*}
&\textrm{let} \;\; BT = [int, real, bool, char]
&
&\textrm{let} \;\; BL = [[BL], BT, TUPLE, struct<TUPLE>]
\\\\
&\textrm{let} \;\; TUPLE = (BT, BT^+)
\end{align*}
\subsubsection{Arithmetic Expressions}
\begin{align*}
&\inference[$EXPR_{int,int}$]{E \vdash e_1 : int \\
                       E \vdash e_2 : int}
                    {E \vdash e_1 \mathbin{\textrm{AOP}} e_2 : int}
&
&\inference[$EXPR_{int,real}$]{E \vdash e_1 : int \\
                       E \vdash e_2 : real}
                    {E \vdash e_1 \mathbin{\textrm{AOP}} e_2 : real}
\\\\
&\inference[$EXPR_{real,int}$]{E \vdash e_1 : real \\
                       E \vdash e_2 : int}
                    {E \vdash e_1 \mathbin{\textrm{AOP}} e_2 : real}
&
&\inference[$EXPR_{real,real}$]{E \vdash e_1 : real \\
                       E \vdash e_2 : real}
                    {E \vdash e_1 \mathbin{\textrm{AOP}} e_2 : real}
\\\\  
&\inference[$ROOT_{int,int}$]{E \vdash e_1 : int \\
                       E \vdash e_2 : int}
                    {E \vdash e_1 \mathbin{\#} e_2 : real}
&
&\inference[$ROOT_{int,real}$]{E \vdash e_1 : int \\
                       E \vdash e_2 : real}
                    {E \vdash e_1 \mathbin{\#} e_2 : real}
\\\\
&\inference[$ROOT_{real,int}$]{E \vdash e_1 : real \\
                       E \vdash e_2 : int}
                    {E \vdash e_1 \mathbin{\#} e_2 : real}
&
&\inference[$ROOT_{real,real}$]{E \vdash e_1 : real \\
                       E \vdash e_2 : real}
                    {E \vdash e_1 \mathbin{\#} e_2 : real}
\end{align*}

\subsubsection{Boolean Expressions}
\begin{align*}
&\inference[$BOOL_{BAOP-int,int}$]{E \vdash e_1 : int \\
                       E \vdash e_2 : int}
                    {E \vdash e_1 \mathbin{\textrm{BAOP}} e_2 : bool}
&
&\inference[$BOOL_{BAOP-int,float}$]{E \vdash e_1 : int \\
                       E \vdash e_2 : float}
                    {E \vdash e_1 \mathbin{\textrm{BAOP}} e_2 : bool}
\\\\
&\inference[$BOOL_{BAOP-float,int}$]{E \vdash e_1 : float \\
                       E \vdash e_2 : int}
                    {E \vdash e_1 \mathbin{\textrm{BAOP}} e_2 : bool}
&
&\inference[$BOOL_{BAOP-float,float}$]{E \vdash e_1 : float \\
                       E \vdash e_2 : float}
                    {E \vdash e_1 \mathbin{\textrm{BAOP}} e_2 : bool}
\\\\
&\inference[$BOOL_{EQUALS}$]{E \vdash e_1 : BL \\
                       E \vdash e_2 : BL}
                    {E \vdash e_1 = e_2 : bool}
&
&\inference[$BOOL_{NEQUALS}$]{E \vdash e_1 : BL \\
                       E \vdash e_2 : BL}
                    {E \vdash e_1 != e_2 : bool}
\\\\
&\inference[$BOOL_{BBOP}$]{E \vdash e_1 : bool \\
                       E \vdash e_2 : bool}
                    {E \vdash e_1 \mathbin{\textrm{BBOP}} e_2 : bool}
\end{align*}
\subsubsection{Conditional Expressions}
\begin{align*}
&\inference[$IF$]{E \vdash b : bool \\
                  E \vdash e : BL}
                 {E \vdash \mathbin{\textrm{if}} \; (b) \; {e}: void}
&
&\inference[$IF-ELSE$]{E \vdash b : bool \\
                  E \vdash e_1 : BL \\
                  E \vdash e_2 : BL'}
                 {E \vdash \mathbin{\textrm{if}} \; (b) \; {e_1} \mathbin{\textrm{else}} {e_2}: void}
\end{align*}

\subsubsection{Loop Expressions}
\begin{align*}
&\inference[$WHILE$]{E \vdash b : bool \\
                  E \vdash e : BT}
                 {E \vdash \mathbin{\textrm{while}} \; (b) \; {e}: void}
&
&\inference[$FOR$]{E \vdash l : [BT] \\
                  E \vdash e : BT'}
                 {E \vdash \mathbin{\textrm{for}} \; \mathbin{\textrm{x}} \; \mathbin{\textrm{in}} \; {l} \; \{e\}: void	| E \vdash x : BT}
\\\\
&\inference[$FOR$]{E \vdash l : [BT] \\
                  E \vdash e : BT' \\
									E \vdash y : int}
                 {E \vdash \mathbin{\textrm{for}} (x,y) \mathbin{\textrm{in}} \; {l} \; \{e\}: void	| E \vdash x : BT}
\end{align*}