\subsection{Functions}
\label{subsec:functions}

Functions can be declared in two ways. By separating the type signature and the function body or by combining the signature and the function body.

\paragraph{Separated function declaration}

The syntax for the separated function declaration is as follows. The type signature and the body must be declared in the same lexical scope.

\begin{verbatim}
  <funcName> : <typeSignature>;
  let <funcName>(<parameterList>) := {<body>};
\end{verbatim}

A concrete example:

\begin{verbatim}
  plus : int -> int -> Int;
  let plus(x, y) := {x + y};
\end{verbatim}


\paragraph{Combined function declaration}

The syntax for the combined function declaration is as follows.

\begin{verbatim}
  let <funcName>(<parameterList>) : <typeSignature> := {<body>};
\end{verbatim}

A concrete example:

\begin{verbatim}
  let plus(x, y) : Int -> Int -> Int := {x + y};
\end{verbatim}
