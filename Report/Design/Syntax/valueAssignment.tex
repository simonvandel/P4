\subsection{Value Assignment}
\label{sec:value_assignment}

A value can be assigned to a symbol in either a constant binding or a variable binding. 

\paragraph{Constant Binding}

A constant binding can never have its value changed. For example, if \enquote{a} is bound to \enquote{5}, \enquote{a} can never refer to another value than \enquote{5} in the same lexical scope.

The syntax for constant value assignment is as follows.
\begin{verbatim}
  let <symbolName> : <type> := <value>
\end{verbatim}

A concrete example:

\begin{verbatim}
  let x : int := 2
\end{verbatim}

\paragraph{Variable Binding}

A variable binding can always change the value it refers to. For example, if \enquote{b} is bound to \enquote{2}, it is perfectly possible to later in the source code refer to {10}.

The syntax for variable value assignment is as follows.
\begin{verbatim}
  var <symbolName> : <type> := <value>

  // a later reassignment
  <symbolName> := <value>
\end{verbatim}

A concrete example:

\begin{verbatim}
  var a : int := 2
  
  // a later reassignment
  a := 2
\end{verbatim}