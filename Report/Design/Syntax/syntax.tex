\subsection{Syntax and Semantics}
\label{sec:syntax}
In this section, will present and discuss some considerations made regarding the syntax and semantics of this language. The aim of this section is to provide insight into the constructs of the language, why they have taken their form.

The primitive data types and operators will be described in \cref{subsec:primitives,subsec:primitiveOps}. After this, functions and their declarations are explained in \cref{subsec:functions}. Next, comments will be briefly mentioned in \cref{subsec:comments}, followed by assignment, or binding, of values to symbols in \cref{subsec:value_assignment}. After this, more complex syntactical constructs, such as actors, structures, loops and selective control structures, will be explained in \cref{subsec:actors,subsec:structs,subsec:forLoopStatements,subsec:ifStatements}.

\subsubsection{Primitive Operators}
\label{subsec:primitiveOps}

Primitive operations are the operations that are build into the language.

\paragraph{Mathematical Operators}
\label{sec:mathOps}

The following mathematical operations are built into the language.

\begin{itemize}
  \item \textbf{+} A binary operator that adds two numbers of the same type
  \item \textbf{-} A binary operator that subtracts two numbers of the same type
  \item \textbf{/} A binary operator that divides two numbers of the same type
  \item \textbf{\%} A binary operator that returns the remainder of a floored division of two numbers of the same type
  \item \textbf{\^}  A binary operator that lifts the first number to the power of the second number
  \item \textbf{\#} A binary operator that roots the first operand to the second operand
\end{itemize}

\sinote{Vi skal nok forklare hvilke nummer typer der kan plusses sammen osv.}

\paragraph{Logical Operations}
\label{sec:logicOps}

\begin{itemize}
  \item \textbf{=} A binary operator that compares the two operands for equality. Returns true if equal. False otherwise.
  \item \textbf{!=} A binary operator that compares the two operands for equality. Returns true if not equal. False otherwise.
  \item \textbf{<} A binary operator that compares the two operands. Returns true if the first operand is strictly less than the second operand. False otherwise.
  \item \textbf{<=} A binary operator that compares the two operands. Returns true if the first operand is less than or equal to the second operand. False otherwise.
  \item \textbf{>} A binary operator that compares the two operands. Returns true if the first operand is strictly greater than the second operand. False otherwise.
  \item \textbf{>=} A binary operator that compares the two operands. Returns true if the first operand is greater than or equal to the second operand. False otherwise.
\end{itemize}

\paragraph{Boolean Operations}
\label{sec:boolOps}

The following operations only operate on bool types.

\begin{itemize}
  \item \textbf{AND} A binary operator that returns true if both values are true. False otherwise
  \item \textbf{OR} A binary operator that returns true if at least one value is true. False otherwise
  \item \textbf{XOR} A binary operator that returns true if only one operand is true. False otherwise
  \item \textbf{NOR} OR negated.
  \item \textbf{NOT} A unary operator that returns the opposite value of the operand.
  \item \textbf{NAND} A binary operator that returns true if none or a single operand is true. False otherwise
\end{itemize}


\subsection{Arithmetic Expressions}
\newcommand{\Twedge}{\mathbin{^\wedge{}}}

\begin{align*}
&\inference[NUM]{}
                  {e \vdash n \Rightarrow_A v}
                  {, \mathcal{N}(n) = v}
\\\\
&\inference[$\text{INVOKE}_A$]{\Braket{S_1,e[p_1' \mapsto \Braket{S_2,p_2'}]} \Rightarrow_A v}
                  {\Braket{p_1(p_2),e} \Rightarrow_A v}
\\                  
&                 {,e(p_1) = \Braket{S_1,p_1'}, e(p_2) = \Braket{S_2,p_2'}}
\\\\
&\inference[$\text{PARENS}_\text{A}$]{e \vdash a_1 \Rightarrow_A a_1'}
                       {e \vdash (a_1) \Rightarrow_A (a_1')}
&
&\inference[$\text{PARENS}_\text{V}$]{}
                       {(v) \Rightarrow_A v}
\\\\
&\inference[$\text{ADD}_\text{L}$]{e \vdash a_1 \Rightarrow_A a_1'}
                    {e \vdash  a_1 + a_2 \Rightarrow_A a_1' + a_2}
&
&\inference[$\text{ADD}_\text{R}$]{e \vdash a_1 \Rightarrow_A a_1'}
                    {e \vdash a_2 + a_1 \Rightarrow_A a_2 + a_1'}
\\\\
&\inference[$\text{ADD}_\text{V}$]{}
                    {v_1 + v_2 \Rightarrow_A v}
                    {, v_1 + v_2 = v}
\\\\
&\inference[$\text{SUB}_\text{L}$]{e \vdash a_1 \Rightarrow_A a_1'}
                    {e \vdash a_1 - a_2 \Rightarrow_A a_1' - a_2}
&
&\inference[$\text{SUB}_\text{R}$]{e \vdash a_1 \Rightarrow_A a_1'}
                    {e \vdash a_2 - a_1 \Rightarrow_A a_2 - a_1'}
\\\\
&\inference[$\text{SUB}_\text{V}$]{}
                    {v_1 - v_2 \Rightarrow_A v}
                    {, v_1 - v_2 = v}
\end{align*}
\begin{align*}
&\inference[$\text{MULT}_\text{L}$]{e \vdash a_1 \Rightarrow_A a_1'}
                     {e \vdash a_1 * a_2 \Rightarrow_A a_1' * a_2}
&
&\inference[$\text{MULT}_\text{R}$]{e \vdash a_1 \Rightarrow_A a_1'}
                     {e \vdash a_2 * a_1 \Rightarrow_A a_2 * a_1'}
\\\\
&\inference[$\text{MULT}_\text{V}$]{}
                     {v_1 * v_2 \Rightarrow_A v}
                     {, v_1 * v_2 = v}
\\\\
&\inference[$\text{DIV}_\text{L}$]{e \vdash a_1 \Rightarrow_A a_1'}
                    {e \vdash a_1 / a_2 \Rightarrow_A a_1' / a_2}
&
&\inference[$\text{DIV}_\text{R}$]{e \vdash a_1 \Rightarrow_A a_1'}
                    {e \vdash a_2 / a_1 \Rightarrow_A a_2 / a_1'}
\\\\
&\inference[$\text{DIV}_\text{V}$]{}
                    {v_1 / v_2 \Rightarrow_A v}
                    {, \frac{v_1}{v_2} = v}
\\\\
&\inference[$\text{MOD}_\text{L}$]{e \vdash a_1 \Rightarrow_A a_1'}
                    {e \vdash a_1 \% a_2 \Rightarrow_A a_1' \% a_2}
&
&\inference[$\text{MOD}_\text{R}$]{e \vdash a_1 \Rightarrow_A a_1'}
                    {e \vdash a_2 \% a_1 \Rightarrow_A a_2 \% a_1'}
\\\\
&\inference[$\text{MOD}_\text{V}$]{}
                    {v_1 \% v_2 \Rightarrow_A v}
                    {, v_1 \;\; \textrm{mod} \;\; v_2 = v}
\\\\
&\inference[$\text{POW}_\text{L}$]{e \vdash a_1  \Rightarrow_A a_1'}
                    {e \vdash a_1 \Twedge a_2 \Rightarrow_A a_1' \Twedge a_2}
&
&\inference[$\text{POW}_\text{R}$]{e \vdash a_1 \Rightarrow_A a_1'}
                    {e \vdash a_2 \Twedge a_1 \Rightarrow_A a_2 \Twedge a_1'}
\\\\
&\inference[$\text{POW}_\text{V}$]{}
                    {v_1 \Twedge v_2 \Rightarrow_A v}
                    {, v_1 ^ {v_2} = v}
\\\\
&\inference[$\text{ROOT}_\text{L}$]{e \vdash a_1 \Rightarrow_A a_1'}
                    {e \vdash a_1 \# a_2 \Rightarrow_A a_1' \# a_2}
&
&\inference[$\text{ROOT}_\text{R}$]{e \vdash a_1 \Rightarrow_A a_1'}
                    {e \vdash a_2 \# a_1 \Rightarrow_A a_2 \# a_1'}
\\\\
&\inference[$\text{ROOT}_\text{V}$]{}
                    {v_1 \# v_2 \Rightarrow_A v}
                    {, \sqrt[v_1]{v_2} = v}
\end{align*}
\subsubsection{Functions}
\label{subsec:functions}

Functions can be declared in two ways. By separating the type signature and the function body or by combining the signature and the function body.

\paragraph{Separated function declaration}

The syntax for the separated function declaration is as follows. The type signature and the body must be declared in the same lexical scope.

\begin{verbatim}
  <funcName> : <typeSignature>;
  let <funcName>(<parameterList>) := {<body>};
\end{verbatim}

A concrete example:

\begin{verbatim}
  plus : int -> int -> Int;
  let plus(x, y) := {x + y};
\end{verbatim}


\paragraph{Combined function declaration}

The syntax for the combined function declaration is as follows.

\begin{verbatim}
  let <funcName>(<parameterList>) : <typeSignature> := {<body>};
\end{verbatim}

A concrete example:

\begin{verbatim}
  let plus(x, y) : Int -> Int -> Int := {x + y};
\end{verbatim}

\subsection{Comments}
\label{sec:comments}

Comments are declared in C style by \enquote{//} being a single line comment and \enquote{/**/} being a multi line comment.

A concrete example:

\begin{verbatim}
  // this is a single line comment
  /* this is
     a
     multi line comment */
\end{verbatim}

\subsection{Initialisations}\label{subsec:initialisations}
A new symbol in TLDR can be created via an initialisation. A symbol always maps to at least one statement, i.e. a symbol can not map to nothing. This means that the environment can never map a symbol to the empty set $\epsilon$. For that reason the initialisation always include an assignment with statements and/or expressions. Since symbols only can map to statements a symbol can only have a value when evaluated.

In traditional mathematical notation, the \enquote{=} symbol is also sometimes used to let certain symbols represent a more complex meaning, in order to simplify something, such as an equation or a function. When used like this, often mathematicians put the word \enquote{let} in front of a statement to denote that it is a definition. We wanted to follow this construct as well letting immutable assignment be denoted in this fashion, since they are comparable to definitions. 

But since we wanted assignments, be it mutable or immutable, to have a similarities, we chose \enquote{:=} for all assignments. This concept is less known in traditional mathematics, but is widely used in computational science. In the historically significant languages Fortran and C, the \enquote{=} symbol, was used for this. However, since we wish to keep that symbol closer to its original meaning, we needed something else. \enquote{:=} was chosen, since it is a known symbol from other languages. The asymmetry of the \enquote{:=} symbol also illustrates that it matters which side of the symbol a variable is on, as opposed to the \enquote{=} symbol.

When assigning statements we differentiates between functions and constants/variables. Whether the symbol is constant or not and whether or not it is a function is defined at initialisation. Whether it is a constant or variable binding is denoted with the keywords \enquote{let}, a constant binding, and \enquote{var}, a variable binding. If the statements are bound constant they can never be changed. This is useful, especially in mathematics where many things both functions and constants never changes. But for things like results, state and generic behaviour a variable binding is useful.

Whether a symbol is a function or not is denoted via parentheses, parentheses meaning that it is a function and no parentheses meaning that it is not. Note that a symbol can have empty parentheses and still be a function meaning that there is a difference between \enquote{let a():int := \dots} and \enquote{let a:int := \dots}. The formal syntax is as follows: 

\begin{grammar}
<Initialisation> ::= ('let' | 'var') (<FuncDecl> | <SymDecl>) ':=' <Expression>

<FuncDecl> -> <Identifier> '(' <Ids>? ')' ':' <Types>

<SymDecl> -> <Identifier> ':' <Types>
\end{grammar}
If a symbol is a constant or variable the right hand side of the \enquote{:=} is evaluated and the symbol is assigned the simplest statement that will always evaluate to the value of the evaluation. If the right hand side is evaluated to a number it is formally described as follows:
\begin{align*}
&\inference[$\text{INIT}_{SYM-A1}$]{\Braket{x,env_s} \Rightarrow_A \Braket{x',env_s}}
                         {\Braket{\Tlet \Ta := x,env_s} \Rightarrow_S \Braket{\Tlet \Ta := x',env_s}}
\\\\
&\inference[$\text{INIT}_{SYM-A2}$]{\Braket{x,env_s} \Rightarrow_A v}
                         {\Braket{\Tlet \Tx := x,env_s} \Rightarrow_S env_s'}
\\
&{\Twhere env_s' = env_{s}[\Tx \mapsto \Braket{\{n\},\epsilon}], \mathcal{N}(n) = v}
\end{align*}
The value the expression evaluates is converted via the $\mathcal{N}$ to the numeral and the symbol is assigned this as within a block. In this way an invocation of the symbol will evaluate the block and the same value each time. If the right hand side is evaluated to a boolean the formal semantics are described as follows:
\begin{align*}
&\inference[$\text{INIT}_{SYM-BOOL}$]{\Braket{b,env_s} \Rightarrow_B \Braket{b',env_s}}
                         {\Braket{\Tlet \Ta := b,env_s} \Rightarrow_S \Braket{\Tlet \Ta := b',env_s}}
\\\\
&\inference[$\text{INIT}_{SYM-\top}$]{\Braket{b,env_s} \Rightarrow_B \top}
                         {\Braket{\Tlet \Tx := x,env_s} \Rightarrow_S env_s'}
\\
&{\Twhere env_s' = env_{s}[\Tx \mapsto \Braket{\{\Ttrue\},\epsilon}]}
\\\\
&\inference[$\text{INIT}_{SYM-\bot}$]{\Braket{b,env_s} \Rightarrow_B \bot}
                         {\Braket{\Tlet \Tx := x,env_s} \Rightarrow_S env_s'}
\\
&{\Twhere env_s' = env_{s}[\Tx \mapsto \Braket{\{\Tfalse\},\epsilon}]}
\end{align*}
Again the resulting value is placed as the statement \enquote{true} or \enquote{false} inside a block, thereby always evaluating to this value.

For the representation of functions in our language, we wanted to stay as close as possible to the mathematical functions, such as \enquote{f(x,y) = z}. In order to achieve this, the type of the function arguments had to either be inherent or declared elsewhere. Since we chose to delimit ourselves from type inherence it must be declared. This is done in the same style as any type declaration, with a kolon. Functions in TLDR are treated like values, being reassignable and potentially having a function take another function as a parameter or give it as a return value.

If a symbol is initialised as a function it is dynamically scoped. This means that the statements, a symbol maps to, is evaluated when it is invoked, in the environment in which it is invoked, meaning that if the function is impure, see \cref{subsubsec:invocation}, it will use the statements the symbols maps to at the time of invocation not initialisation. This means that a function when not invoked maps to the statements it was assigned and other functions can be assigned these or use them as inputs.

\begin{align*}
&\inference[$\text{INIT}_{FUNC1}$]{}
                         {\Braket{\Tlet \Tx() := \{S\},sEnv} \Rightarrow_S sEnv'}
												 {, sEnv' = sEnv[\Tx \mapsto \Braket{S,\epsilon}]}
\end{align*}
If the symbol is a function and takes input parameters these are placed as formal parameters in the symbols ordered set the symbol maps to in the order they appear.
\begin{align*}
&\inference[$\text{INIT}_{FUNC2}$]{}
                         {\Braket{\Tlet \Tx(x) := \{S\},sEnv} \Rightarrow_S sEnv'}
												 {, sEnv' = sEnv[\Tx \mapsto \Braket{S,y}]}
\end{align*}
Expanding the amount of input parameters is a trivial task. All symbols are placed in the set in the order they appeal in the code. 
\begin{align*}
&\inference[$\text{INIT}_{FUNC3}$]{}
                         {\Braket{\Tlet \Tx(x,y) := \{S\};,sEnv} \Rightarrow_S sEnv'}
												 {, sEnv' = sEnv[\Tx \mapsto \Braket{S,\{x,y\}}]}
\end{align*}
Note that all rules are defined as constant bindings (\enquote{let}) but the exact same semantic goes for all variable bindings, since the allowance of reassignment is checked by the type system.

\begin{grammar}
<Identifier> ::= 'me'
 \alt <Id>
 \alt <Id> <Accessor>

<Id> ::= [a-zA-Z][a-zA-Z\_0-9]*-('let' | 'var' | 'int' | 'real' | 'char' | 'bool' | 'struct' | 'actor' | 'receive' | 'send' | 'spawn' | 'return' | 'for' | 'in' | 'if' | 'else' | 'while' | 'die' | 'me ')

<Ids> ::= <Identifier> (',' <Identifier>)*

<Types> ::= <Type> ('->' <Type>)*

<Type> ::= <PrimitiveType>
 \alt <Identifier>
 \alt <ListType>
 \alt <TupleType>

<TupleType> ::= '(' <Types> ')'

<ListType> ::= '\{' <Types> '\}'

<PrimitiveType> ::= 'int' | 'real' | 'char' | 'bool'
\end{grammar}

As can be seen, values can either be an immutable constant, or a muteable variable.

A constant binding can never have its value changed. For example, if \enquote{a} is bound to \enquote{5}, \enquote{a} can never refer to another value than \enquote{5} in the same lexical scope. 

The syntax for constant value assignment is as follows.
\begin{verbatim}
  let <symbolName> : <type> := <value>
\end{verbatim}
A concrete example:

\begin{verbatim}
  let x : int := 2
\end{verbatim}

A variable binding can always change the value it refers to. For example, if \enquote{b} is bound to \enquote{2}, it is perfectly possible to later in the source code refer to {10}.

The syntax for variable value assignment is as follows.

\begin{verbatim}
  var <symbolName> : <type> := <value>

// a later reassignment
  <symbolName> := <value>
\end{verbatim}
A concrete example:

\begin{verbatim}
  var a : int := 2
  
// a later reassignment
  a := 2
\end{verbatim}

\subsection{Assignment}\label{subsec:assignment}
\begin{align*}
&\inference[ASS]{\Braket{S,e} \Rightarrow_S \Braket{S',e'}}
                 {\Braket{\Tx := S,e} \Rightarrow_S \Braket{\Tx := S',e'}}
\\\\
&\inference[ASS]{\Braket{S,e} \Rightarrow_S v}
                 {\Braket{\Tx := S,e} \Rightarrow_S e'}
								 {, e' = e[\Tx \mapsto \Braket{\{n\},\epsilon}], \mathcal{N}(n) = v}
\end{align*}


\begin{align*}
\intertext{}
&\inference[LET]{}
                 {\Tenv \mathbin{\text{let}} \; e:T: ok}
\\\\
&\inference[VAR]{}
                 {\Tenv \mathbin{\text{var}} \; e:T: ok}
\\\\
&\inference[LET1]{\Tenv e_1: \Tt & \Tenv e_2: \Tt}
                 {\Tenv \mathbin{\text{let}} \; e_1 := e_2: ok}
\\\\
&\inference[VAR1]{\Tenv e_1: \Tt & \Tenv e_2: \Tt}
                 {\Tenv \mathbin{\text{var}} \; e_1 := e_2: ok}
\end{align*}

\section{Actors}

Here follows descriptions of the usage of actors in TLDR. This includes different principles, functionalities, syntactical and the semantics. But before we can discuss the use of actors in TLDR, we must first cover the semantics of the parallelism used.

\subsubsection{Semantics of parallelism}

\kanote{forklaring af parallisms semantic}

\begin{align*}
%%%%%%%%%%%%%%%%%%%%%%%%%%%%%%%%%%%%%%%%%%%%%%%%%%%%%%%%%%%%%%%
\intertext{The left side of a parallel statement is executed, but not finished. The environment and actor model is updated when executing $S_1$.}
&\inference[$\text{PAR}_1$]{\Braket{S_1,e_1,\Ta} \Rightarrow_S \Braket{S_1',e_1',\Ta'}} 
                           {\Braket{S_1,e_1,\Ta}|\Braket{S_2,e_2,\Ta} \Rightarrow_S \Braket{S_1',e_1',\Ta'}|\Braket{S_2,e_2,\Ta'}}
%%%%%%%%%%%%%%%%%%%%%%%%%%%%%%%%%%%%%%%%%%%%%%%%%%%%%%%%%%%%%%%
\intertext{The left side of a parallel statement is executed, and finishes. The environment and actor model is updated when executing $S_1$.}
&\inference[$\text{PAR}_2$]{\Braket{S_1,e_1,\Ta} \Rightarrow_S \Braket{e_1',\Ta'}} 
                           {\Braket{S_1,e_1,\Ta}|\Braket{S_2,e_2,\Ta} \Rightarrow_S \Braket{S_2,e_2,\Ta'}}
%%%%%%%%%%%%%%%%%%%%%%%%%%%%%%%%%%%%%%%%%%%%%%%%%%%%%%%%%%%%%%%
\intertext{The right side of a parallel statement is executed, but not finished. The environment and actor model is updated when executing $S_2$.}
&\inference[$\text{PAR}_3$]{\Braket{S_2,e_2,\Ta} \Rightarrow_S \Braket{S_2',e_2',\Ta'}} 
                           {\Braket{S_1,e_1,\Ta}|\Braket{S_2,e_2,\Ta} \Rightarrow_S \Braket{S_1,e_1,\Ta'}|\Braket{S_2',e_2',\Ta'}}
%%%%%%%%%%%%%%%%%%%%%%%%%%%%%%%%%%%%%%%%%%%%%%%%%%%%%%%%%%%%%%%
\intertext{The right side of a parallel statement is executed, and finishes. The environment and actor model is updated when executing $S_2$.}
&\inference[$\text{PAR}_4$]{\Braket{S_2,e_2,\Ta} \Rightarrow_S \Braket{e_2',\Ta'}}
                           {\Braket{S_1,e_1,\Ta}|\Braket{S_2,e_2,\Ta} \Rightarrow_S \Braket{S_1,e_1,\Ta'}}
%%%%%%%%%%%%%%%%%%%%%%%%%%%%%%%%%%%%%%%%%%%%%%%%%%%%%%%%%%%%%%%
\end{align*}

\subsubsection{Isolation and Independence}

A central principle in TLDR is the use of actors, based on the actor model. Actors are to be seen as entities with interaction. In other words, in order for a construct to qualify as an actor, it must define a way to behave when other actors interact with it. Actors should function independently, and in that regard, not be open to direct manipulation and only able to be changed through the messages it receives. This requirement is due to the wish of separation of processes, which will allow for greater concurrency by letting processes operate on local data instead of global, shared data. Therefore, TLDR tries to encourage natural isolation of functionality through actors, which in turn will also give a greater control of race conditions as no data is ever accessed by more than one process.

\subsubsection{The Main Actor}

Main is always the first actor, and any program writtin in TLDR is started with main receiving the arguments message. This is done to force the programmer to start and end the program with an actor, which will better support the actor modeling perspective. This means that the main actor is started with a message for the arguments, and when the main actor is killed the program will stop executing, whether or not there are still working actors.

The argument message called args, is a struct which consists of a argument counter called argv, and a lists of char lists called argv.

Another way that this affects the programmer, is the idea of spawning actors and having them send messages to main, instead of calling functions and having them return. This also means that there must be a way to reference main, since it is not spawned by another actor. This is solved by the introduction of the \enquote{me} keyword, which will evaluate to a reference to the current actor. This is very useful, especially if an actor wishes to delegate work to other actors. The message that is sent with the work, must simply contain a reference back to the delegating actor, which was included through the use of the \enquote{me} keyword.

Here are the semantics for the main actor.

\begin{align*}
&\inference[$\text{MAIN}$]{input \mapsto \Braket{S,\epsilon}}
                          {\Braket{\Treceive \Tr:args := \{S\},e} \Rightarrow_S \Braket{\Tr := input;S, e]}}
\end{align*}

\subsubsection{Construction of an Actor}
\label{sub:constructionOfAnActor}

The syntactical declaration of an actor is as follows:

\begin{lstlisting}
actor <identifier> := {
  <functionality>
}
\end{lstlisting}

And a concrete example could be:

\begin{verbatim}
  actor earth := {
    var temperature:real := 0;
    receive sunlight:light := {
      temperature := temperature + 0.1;
    }
  }
\end{verbatim}

As shown, the actor keyword precedes the definition, denoting the meaning of said definition. After the keyword an identifier of the declaration is needed, which will serve as the specific actor type. It is suggested that this identifier reflects the role of the actor in a context of use. Noticeably there is no \enquote{let} or \enquote{var} keyword in front of the definition, as there usually is when assigning. This is a deliberate choice since \enquote{let} and \enquote{var} implies interchangeability, which is not an option is this case. If variable declaration of actor definitions were possible, it would effectively be the equivalent of changing the definition of a type on run-time, which would make little sense, and completely undermine the type safety in the language.

Some general semantics of actors:

\begin{align*}
&\inference[TYPEOF]{}
                  {e \vdash m \Rightarrow_T t}
                  {, \mathbb{T}(m) = t}
\\\\
&\inference[$\text{ACTOR}$]{a' = a[\Tact \mapsto e \times st]}
                           {\Braket{\Tactor \Tact := \{S\}, a} \Rightarrow_S \Braket{S,e,at[\Tact \mapsto S],a'}}
\\\\
&\inference[$\text{ACTOR}$]{}
                           {\Braket{\Tactor \Tact := 1S, at} \Rightarrow_S \Braket{at[\Tact \mapsto S]}}
\end{align*}

\subsubsection{Basic Actor Functionality}
\label{subsubsec:BasicActorFunctionality}
There are four basic functionalities for actors: \enquote{spawn}, \enquote{die}, \enquote{send}, and \enquote{receive}, which are all used through keywords. It was desired to keep the syntax of these functionalities different from the syntax of functions. Even though they behave much like functions, taking input and giving output, they are more powerful. For example, regular functions cannot contain a type as a parameter, but the \enquote{spawn} functionality does this. Due to this and more differences, which follow below, it was decided to separate them syntactically.

The \enquote{spawn} functionality is used to create new instances of actors. And example could be:

\label{actorfuncSpawn}
\begin{lstlisting}
actor <identifier> := {
 <functionality>
}

let MyActor:<identifier> := spawn <identifier> <message>;

or alternatively:

var MyActor:<identifier> := spawn <identifier> <message>;
\end{lstlisting}

As can be seen above, there are four parts of the spawn functionality: an identifier, the keyword, the type of the actor, and optionally an initial message. Firstly, the identifier is preceded by a keyword for mutability, such as \enquote{let} or \enquote{var}. This allows for the substitution of handles, which provides possibilities of dynamic changes. This however opens up the possibility of \enquote{losing contact} with an actor, if the handle is replaced. This could potentially lead to memory leaks if not handled properly. \kanote{reference til garbage collection afsnit}

When spawning a new actor, you can also choose to add a message. The reason for this is to give the programmer a way of initialising the new actor with a certain message. In object-oriented languages this is usually done with a constructor, however doing it via a constructor would conflict with a central principle, since it would mean manipulating an actors state directly.

The semantics of \enquote{spawn} is as follows:

\begin{align*}
&\inference[$\text{SPAWN}$]{}
                       {\Braket{\Tlet \Tact:T := \Tspawn \; T \Tm,\Ta} \Rightarrow_S \Braket{\Tsend \Tact \Tm,\Ta[act \mapsto e \times st]}}
\\
&                       {, \Ta(x) \mapsto \Braket{S,p} , \Ta' = \Ta[\Tact \mapsto \Braket{S, p}]}
\end{align*}

Type rules for \enquote{spawn}:

\begin{align*}
\intertext{}
&\inference[SPAWN]{}
                 {\Tenv \mathbin{\text{spawn}} \; \Tt: \Tt}
\end{align*}

After an actor has been spawned, it will be possible to send messages to it. This is done with the \enquote{send}-keyword. This can be done as follows:

\label{actorfuncSend}
\begin{lstlisting}
MyMsg:int := 42;

Send MyActor MyMsg;
\end{lstlisting}

It is also possible for actors to send messages to themselves by using the \enquote{me}-keyword. Such messages will be treated the same as any other message.

The semantics of \enquote{send} is as follows:

\begin{align*}
&\inference[$\text{SEND}$]{e_2 = a(act),m \Rightarrow_T t}
                       {\Braket{\Tsend \Tact \Tm ; S,\Ta,e_1} \Rightarrow_S \Braket{S,e_1,a}|\Braket{\_t(m),e_2,a}}
\end{align*}

and the type rules for \enquote{send}:

\begin{align*}
\intertext{}
&\inference[SEND]{\Tenv m : \Tt & \Tenv a: \Tt'}
                 {\Tenv \mathbin{\text{send}} \; \mathbin{\text{m}} \; \mathbin{\text{a}} : ok }
\end{align*}

When an actor is sent a message, it must act according to a defined a way of handling that type of message. This definition is declared with the \enquote{receive}-keyword, which creates a method within the actor that is called when the actor receives a corresponding message. The syntax can be seen below:

\label{actorfuncReceive}
\begin{lstlisting}
actor <nameOfActor> := {
 receive <nameOfMessage>:<typeOfMessage> := {
  <functionality>
 }
}
\end{lstlisting}

In this example the receive-method defines the way messages of the type \enquote{<typeOfMessage>} are handled. Within the functionality \enquote{<nameOfMessage>} is the reference to the message. This message is immutable no matter if it was mutable where it was sent from. This is done to discourage further use of old messages.

It is also the intention to include a \enquote{wait on <typeOfMessage>} keyword, which will cause the actor to not evaluate the next message in the messagequeue, but instead traverse the queue, until a message matching \enquote{<typeOfMessage} is found. Then that message is de-queued and evaluated. This has not been implemented in the current release of TLDR, but it will be a central part of supporting discrete simulations.

The semantics of \enquote{receive} is as follows:

\begin{align*}
&\inference[$\text{RECEIVE}$]{}
                           {\Braket{\Treceive r:t := \{S\};,e} \Rightarrow_S \Braket{e[\_t \mapsto \Braket{S,r}]}}
\end{align*}

The type rules for \enquote{receive} are:

\begin{align*}
\intertext{}
&\inference[RECEIVE]{\Tenv m : \Tt}
                 {\Tenv \mathbin{\text{recieve}} \; \mathbin{\text{m}} :\Tt : ok }
\end{align*}

When an actor is no longer needed, it is possible to discard it with the \enquote{die}-keyword. In other languages, using the actor model, \enquote{die}, or similar functionality, is usually called by the parent of the actor, that is, the actor that spawned the actor. In TLDR however, \enquote{die} can only be called by the actor itself. This is done in order to keep the principle of only interacting with actors through messages, which is a simpler way of handling actors in TLDR, since the language does not have a built-in supervisor functionality as described in \cref{actSupervisors}. 

When an actor dies it stops immediately and does not compute further, and whatever messages might have been in the actors message-queue, will be lost. 

\subsubsection{Actor references}

In TLDR declarations must be accomodated by an assignment of a value, since no primitives can have a null value. However, since actors handles have the special position in the language of being the only value which is passed by reference, it should incompass a null value. The reason for this becomes apparent if one considers the following example:

\begin{lstlisting}
actor main := {
  receive arguments:args := {
    var jack:man := spawn man;
  }
}

actor man := {
  var bestMan:man := spawn man;
  }
}
\end{lstlisting}

In this example, the bestMan variable would result in an endless recursive spawn chain of actors. This is the most obvious problem, but there are multiple senarioes where forced value assignment for reference types become problematic. Due to this, the language allows for null references.

\subsubsection{Comparison}

When comparing actors, they are considered equal if they are reference equals.

Here are the type rules for comparison:

\begin{align*}
\intertext{}
&\inference[ACTOR]{\Tenv e_1: \Tt & \Tenv e_2: \Tt}
                 {\Tenv e_1 = e_2: \Tbool}                 
\end{align*}

%mainfile: ../../master.tex
\subsection{Structures}
\label{subsec:structs}

In TLDR there are three ways to do encapsulation. Actors, Tuples and structs. Structs are unique by being fully accessible within the scope, and having named fields. Structs are especially useful in TLDR for creating messages.

\subsubsection{Defining Structures}
\label{sec:defStructures}

\subsubsection{Syntax}

Structures are defined by using the \enquote{struct} keyword. The grammar for declaring structs are as follows:

\begin{grammar}
  <Struct> ::= 'struct' <Identifier> ':= \{' <TypeDecls> '\}'
\end{grammar}

And a concrete example:

\begin{lstlisting}[style=TLDR]
  struct Person := {Name:[char]; Age:int}
\end{lstlisting}

\subsubsection{Semantics}

having these semantics:

\begin{align*}
\intertext{In the case that we have multiple S of assignments, we can rewrite the T to now map to new s' that includes x, in the new st' environment and the rest of declaration statements}
&\inference[$\text{STRUCT}$]{}
                            {\Braket{\Tstruct T := \{x:T';S\}, env_s, st} \Rightarrow_S \Braket{\Tstruct T := \{S\},env_s,st'}}
\\
&{\Twhere st' = st[T \mapsto s'],st(T) = s,s' = s \cup x]}
\\\\
\intertext{In the case that we have multiple S of assignments, we can rewrite the T to now map to new s' that includes x, in the new st' environment}
&\inference[$\text{STRUCT}$]{}
                            {\Braket{\Tstruct T := \{x:T\}, env_s, st} \Rightarrow_S \Braket{env_s,st'}}
\\
&{\Twhere st' = st[T \mapsto s'],st(T) = s,s' = s \cup x]}
\end{align*}

\begin{align*}
&\inference[$\text{STRUCT}$]{}
                            {\Braket{\Tstruct T_1 := (x:T_2;S), env_s} \Rightarrow_S \Braket{\Tstruct T_1 := (S),env_s'}}
\\
&{\Twhere env_s' = env_s[T \mapsto \Braket{\epsilon,s'}],env_s(T) = s,s' = s \cup x}
\\\\
&\inference[$\text{STRUCT}$]{}
                            {\Braket{\Tlet \; \Ta:T := (S),env_s} \Rightarrow_S \Braket{\Ta := (S),env_s'}}
\\
&{env_s' = env_s[x \mapsto (S,s)],env_s(T) = (\epsilon,s)]}
\\\\
&\inference[$\text{STRUCT}$]{\Braket{x,sEnv} \Rightarrow_a \Braket{x',sEnv}}
                            {\Braket{\Tlet \Ta := (f := x;S),env_s} \Rightarrow_S \Braket{\Tlet \Ta := (f := x';S),env_s}}
\\\\
&\inference[$\text{STRUCT}$]{\Braket{x,sEnv} \Rightarrow_a n}
                            {\Braket{\Tlet \Ta := (f := x;S),env_s} \Rightarrow_S \Braket{\Tlet \Ta := (S),env_s[s.f \mapsto {n}]}}
\\\\
&\inference[$\text{STRUCT}$]{sEnv(x) = \Braket{(S),s}}
                            {\Braket{\Tlet \Ta := (f := x;S),env_s} \Rightarrow_S \Braket{\Tlet \Ta.f.s := s;\Ta := (S),env_s}}
\\\\
&\inference[$\text{STRUCT}$]{\Braket{x,sEnv} \Rightarrow_a n}
                            {\Braket{\Tlet \Ta := (f := x;S),env_s} \Rightarrow_S \Braket{\Tlet s := (S),env_s[s.f \mapsto {n}]}}
\\\\
&\inference[$\text{LIST}$]{}
                            {\Braket{s:[T] := [x_n,..,x_m],sEnv_1} \Rightarrow_S \Braket{s:[T] := [x_{n+1},..,x_m],sEnv_1}}
\\
&{sEnv_1(s) = \Braket{\epsilon,\epsilon,sEnv_2},sEnv_2' = sEnv_2[n \mapsto x_n],sEnv_1'(s) = \Braket{\epsilon,\epsilon,sEnv_2'}}
\\\\
&\inference[$\text{TUPLE}$]{}
                           {\Braket{s:[T] := [x_n,..,x_m],sEnv_1} \Rightarrow_S \Braket{s:[T] := [x_{n+1},..,x_m],sEnv_1}}
\\
&{sEnv_1(s) = \Braket{\epsilon,\epsilon,sEnv_2},sEnv_2' = sEnv_2[n \mapsto x_n],sEnv_1'(s) = \Braket{\epsilon,\epsilon,sEnv_2'}}
\end{align*}

\subsubsection{Type Rules}

\begin{align*}
\intertext{The elements can be of any type defined in $\Tt$}
&\inference[STRUCT]{E[s \mapsto (e_1:\Tt_1;e_2:\Tt_2;...;e_n:\Tt_n) \rightarrow ok]\vdash S : ok & }
                 {\Tenv \mathbin{\text{struct s}} := \{e_1:\Tt_1;e_2:\Tt_2;...;e_n:\Tt_n\}; S: ok}
\end{align*}



\subsubsection{Initialising Structures}
\label{sec:initStructures}

Structs can be initialised and assigned to symbols using either a constant assignment or a variable assignment. Structs initialised as a constant assignment cannot change any of the fields of the structs; the struct is immutable. Structs initialised as a variable assignment can change all of its fields at any time; the struct is mutable.

\subsubsection{Syntax}

The syntax for initialising a struct is as follow.

\begin{grammar}
<StructLiteral> ::= '(' (<Reassignment>';')* ')' (':' <Identifier>)?
\end{grammar}


With concrete examples for immutables:

\begin{verbatim}
  let Alice:Person := (Name := "Alice"; Age := 20);
\end{verbatim}

And for mutable struct is as follow.

\begin{verbatim}
  var Alice:Person := (Name := "Alice"; Age := 20);
\end{verbatim}

And for usage in lists.

\begin{verbatim}
  [(Name := "Alice"; Age := 20):Person];
\end{verbatim}

\subsubsection{Access to Structure Fields}
\label{sec:accessStructFields}

Fields can be access using the following syntax.

\begin{verbatim}
  Alice.Name; // "Alice"
\end{verbatim}

Structs declared as mutable can have fields reassigned using the following syntax.

\begin{verbatim}
  Alice.Name; // "Alice"
  Alice.Name := "Bob";
  Alice.Name; // "Bob";
\end{verbatim}
 
\subsubsection{Semantics}

\begin{align*}
\intertext{In the case that we have multiple S of assignments, we can rewrite the s to now having an accessor that maps to value of the first assignment and the rest of pending assignments in s}
&\inference[$\text{STRUCT}$]{}
                            {\Braket{\Tlet \; s:T := (f := x;S),env_s,st} \Rightarrow_S \Braket{s.f := x;\Tlet \; s:T := (S),env_s,st}}
\\\\
\intertext{In the case that we have only one assignment, we can rewrite the s to now having an accessor that maps to value of the assignment}
&\inference[$\text{STRUCT}$]{}
                            {\Braket{\Tlet \; s:T := (f := x),env_s,st} \Rightarrow_S \Braket{s.f := x,env_s,st}}
%\\\\
%&\inference[$\text{STRUCT}$]{}
%                            {\Braket{(f := x;S),env_s,st} \Rightarrow_S \Braket{s.f := x;\Tlet \; s:T := (S),env_s,st}}
%\\\\
%&\inference[$\text{STRUCT}$]{}
%                            {\Braket{(f := x):T,env_s,st} \Rightarrow_S \Braket{s.f := x,env_s,st}}
\end{align*}

\subsubsection{Type Rules}

Each e of type t is matching the declared struct types, evaluates to the type of the declared struct.

\begin{align*}
&\inference[STRUCTLITERAL]{\Tenv (e_1 : \Tt_1;e_2 : \Tt_2;...;e_n : \Tt_n):\Tt'}
                 {\Tenv \mathbin{\text{(}} e_1; e_2;...;e_n\mathbin{\text{)}}:\Tt': \Tt'}
\end{align*}

\subsubsection{Comparison of Structs}
\begin{align*}
&\inference[STRUCT]{\Tenv e_1: \Tt & \Tenv e_2: \Tt}
                 {\Tenv e_1 = e_2: \Tbool}
\end{align*}

\subsubsection{For-loop Statements}
\label{subsec:forLoopStatements}

A for-loop iterates through a list of elements.

The for loop statement has the following syntax.

\begin{verbatim}
  for <element> in <collection> {<loopBody>}
\end{verbatim}

A concrete example:

\begin{verbatim}
  for i:int in [0..10]:[int] { /* Do stuff with i elements */ }
\end{verbatim}
\subsubsection{While-loop Statements}
\label{subsec:whileLoopStatements}

The while loop statement runs a block of statements until a boolean expression returns false.

\paragraph{While loop}
\label{sec:whileLoop}

The while statement has the following syntax.

\begin{verbatim}
  while (<boolExp>) {<loopBody>}
\end{verbatim}

A concrete example:

\begin{verbatim}
  var i:int := 0;
  while (i < 10) {i := i + 1}
\end{verbatim}

\subsubsection{Semantics}

\begin{align*}
&\inference[$\text{WHILE}_\top$]{e \vdash b \Rightarrow_B \top}
                       {\Braket{\Twhile(b)\{S\},e} \Rightarrow_S \Braket{\{S\}; \Twhile (b)\{S\},e}}
\\\\
&\inference[$\text{WHILE}_\bot$]{e \vdash b \Rightarrow_B \bot}
                       {\Braket{\Twhile(b)\{S\},e} \Rightarrow_S e}
\end{align*}
\subsection{If-statements}
\label{subsec:ifStatements}

If-statements can be written either as a if-then-else statement or just as an if-statement. The following shows either way.

\begin{verbatim}
  // if-then-else statement
  if (<condition>) {<what to do if condition is true>}
  else {<what to do if condition is false>}

  // if-statement
  if (<condition>) {<what to do if condition is true>}
\end{verbatim}

A concrete example:

\begin{verbatim}
  // if-then-else statement
  if (2 + 2 = 4) {"math works!"}
  else {"something is wrong here!"}

  // if-statement
  if (remainingTime < 10) {initiateCountdown()}
\end{verbatim}

\subsubsection{Match-statements}
\label{subsec:matchStatements}

Match statements can be seen as syntactical sugar for multiple chained if-statements. The syntax is as follows.

\begin{verbatim}
  match <whatToMatchOn> {
    <case1> -> <actionOnCase1>
    <case2> -> <actionOnCase2>
    ...
    <caseN> -> <actionOnCaseN>
  }
\end{verbatim}

A concrete example:

\begin{verbatim}
  match (1, 2) {
    (0, n) -> // this case will never be reached
    (1, n) -> print("Case reached!");
    _ -> // default case that matches everything
  }
\end{verbatim}
