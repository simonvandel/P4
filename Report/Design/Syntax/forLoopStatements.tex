\subsection{For-loop Statements}
\label{subsec:forLoopStatements}

There are three for-loop statements defined in the language. A traditional C for-loop, a for-in loop and a variation of the for-in loop where both the index and the element of the collection being looped is provided.

\subsubsection{C Inspired For-loop}
\label{sec:cForLoop}

The syntax for the first for-loop variation is defined below.

\begin{verbatim}
  for(<startStatement>; <condition>; <updateStatement>) {<loopBody>};
\end{verbatim}

A concrete example:

\begin{verbatim}
  for(var i:int := 0; i < 10; i := i+1)) {/* Do stuff */};
\end{verbatim}

\subsubsection{For-in loop}
\label{sec:forInLoop}

The for-in loop statement has the following syntax.

\begin{verbatim}
  for <element> in <collection> {<loopBody>}
\end{verbatim}

A concrete example:

\begin{verbatim}
  for i:int in [0..10]:[int] { /* Do stuff with i elements */ }
\end{verbatim}

\subsubsection{For-in loop with index}
\label{sec:forInLoopIndex}

A variation of the for-in loop that also provides the index of the current element, has the following syntax.

\begin{verbatim}
  for (<index>, <element>) in <collection> {<loopBody>}
\end{verbatim}

A concrete example:

\begin{verbatim}
  for (index:int, elem:int) in [0..10]:[int] { 
    /* Do stuff with i elements and index */ 
  }
\end{verbatim}
