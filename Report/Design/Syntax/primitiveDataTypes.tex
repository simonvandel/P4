\subsubsection{Primitive Data Types}
\label{subsec:primitives}

A primitive is a predefined value type that cannot be constructed using other value types.

The following primitives exist in the language. All primitives are lower-cased.

\begin{itemize}
  \item int
  \item real
  \item bool
  \item char
  \item unit/void/empty \sinote{bliv enige om navn. Skal rettes længere nede også}
\end{itemize}

\paragraph{Integer}
\label{subsubsec:int}

The int primitive can have values of 0 to \sinote{hvor mange bits?}. The int primitive's literal representation is as whole numbers e.g. \emph{2}.

\paragraph{Real}
\label{subsubsec:real}

The real primitive can have values of \sinote{hvordan definerer vi det her?}. The real primitive's literal representation is as decimal numbers with fractions e.g. \emph{2.5} or \emph{2.0}.

\paragraph{Bool}
\label{subsubsec:bool}

The bool primitive's value can be either \emph{true} or \emph{false}.

\paragraph{Char}
\label{sec:char}

The char primitive can have values of the ASCII standard. It is written literally as a character defined in the ASCII standard, surrounded by single quotation marks. For example: \emph{ '0' } and \emph{ 'A' }.

\paragraph{Unit/void/empty}
\label{sec:unit/empty/void}

The unit/void/empty primitive can have only one value: itself. The use of the primitive is to signal emptiness.
