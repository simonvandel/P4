\section{Compiler-Compiler}
\label{sec:compiler_compiler_choice}

It was decided to prioritize the programmers above a simple grammar. This means that keeping the chosen syntax is preferred, and therefore accepting a more complex grammar, should it be needed. Due to this possibility of great complexity it was chosen to use a compiler-compiler. The choice of compiler-compiler was based on a series of criteria, which are listed below with the corresponding reason for the criteria.

\begin{enumerate}
\item \textbf{Language of the compiler} It was chosen to construct the compiler in a language compatible with the common language infrastructure (CLI), since the developers had previous experience with languages running on this. This makes C\# and F\# preferred target languages for the compiler-compiler.

\item \textbf{LR parser} The compiler-compiler should support LR parsing since left recursions have been used in the grammar specifying the language.\\

\item \textbf{Action Code} The parser should support action code and the grammar and action code should preferably be in separate files.\\

\end{enumerate}

Two compiler-compilers matching the criteria list were found. These were HIME and SableCC. SableCC supports LALR parsing, which parses the same set of grammars as LR parser, which makes SableCC a suitable parser. HIME was ultimately chosen as the compiler-compiler because it is frequently updated and targets C\# as the main target language where SableCC targets Java and later on started supporting C\#.
