\section{AST Traversal}

This section describes how the AST is operated on during the different phases of the compiler. As an example, let us look at the code generation phase. The phase is essentially a function that takes an AST and returns a string of characters representing the generated target code. The function must traverse all the nodes in the AST. As seen in \cref{lst:ast_listing}, the AST is recursively defined. The root of the AST is \enquote{Program}, containing a list of ASTs. The traversal is done by using pattern matching on the AST. In the case of the code generation function, if the AST is a \enquote{Program} node, the code generation function is simply applied to all subtrees contained in the data type, and the resulting generated target code is merged together. A simplified example of the pattern matching can be seen in \cref{lst:ast_traversal}.

\begin{lstlisting}[style = fsharp, label = lst:ast_traversal, caption = {Example of traversing AST}]
let rec codeGen (ast:AST) : string =
  match ast with
  | Program stms -> // apply codeGen to all stms and merge result
  | Constant (type, value) -> // generate the code for the constant based on the type and value of the constant.
\end{lstlisting}
