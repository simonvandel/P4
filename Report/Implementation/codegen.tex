\section{Code Generation}

This section describes the code generation phase of the compiler developed in this project. The focus is on the overall aspect of transforming an abstract syntax tree to the various language constructs in TLDR, however not every language construct will have its code generation described. 

The purpose of the phase is to output code that can later be run on the computer as an executable. LLVM was chosen as the target language for this compiler in \cref{sec:targetLanguage}, so this is what the code generation phase will output. This generated LLVM IR is then compiled into native code by using the clang compiler.

\subsection{Data Sizes}

Even though the language design of TLDR requires integers and reals to be of arbitrary precision as seen in \cref{lang_design} \sinote{indsæt reference til afsnit om arbritrary precision}, the code generation phase generates integers and reals of 64-bit size. However, one way of adding arbitrary precision arithmetic to the compiler would be to use a library such as GMP \cite{gmp}, which we chose to do.

\subsection{Actors}

Actors are implemented using the C library \enquote{libactor} \cite{libactor}. Using the library, we can avoid having to deal with implementation of actors ourselves. However, as we have not written the library ourselves, we do not have total control over it. This also means that there might be differences in the semantics between \enquote{libactor} and the semantics for actors as described in \cref{sec:syntax}.

Every actor in TLDR is generated into a LLVM function with the following behaviour outlined in pseudo code.

\begin{lstlisting}[breaklines]
active = true
msg = null
<initialisation of fields in actor>
start:
while (active) do
  msg = actor_receive()
  switch (msg.type) do
  case 101:
    goto recv_kill
  case 102:
    goto recv_102
  case 103:
    goto recv_103

return

recv_102:
<body of first receive>
goto start 
recv_103:
<body of second receive>
goto start 
recv_kill:
active = false
goto start
\end{lstlisting}

A actor simply checks for new messages, and performs the corresponding message, based on the type of the received message, by jumping to a label. This while loops ends when the actor receives a kill message.

By having receive constructs simply be labels, all data of an actor can be contained in a single function.

\subsection{Lists}

Lists in TLDR are simply generated as static LLVM arrays. This mean that their size is known at compile time, and that no elements can be added at run-time. This is currently a limitation in the current compiler, but is something that is doable if the project group had prioritised this.

\subsection{Optimisations For Free}

Even though speed has not been a primary goal of this implementation of TLDR, many optimisations come for free by generating LLVM IR. The code generation implemented in this project is very naive and thous does not do any optimisations in its generation. However, many optimisation passes are developed in the LLVM project. We can apply these optimisations very easy to our naive code generation, simply by running a LLVM optimiser on the LLVM IR generated by the compiler developed in this project.

As an example, let's try and compile the following TLDR code to LLVM IR.

\begin{lstlisting}[breaklines]
actor main := {
  receive arguments:args := {
    let res:int := (2 + 3) * 4 / 2;
    printint(res);
    die;
  };
};
\end{lstlisting}

The code generation generates the following LLVM IR.

\begin{lstlisting}[breaklines]
declare double @llvm.pow.f64(double, double)
declare double @llvm.powi.f64(double, i64)
declare i32 @puts(i8*)
declare i32 @printf(i8*, ...)
declare void @actor_init(...)
declare void @actor_wait_finish(...)
declare void @actor_destroy_all(...)
%struct.actor_message_struct = type { %struct.actor_message_struct*, i64, i64, i64, i8*, i64}
%struct.actor_main = type { i32, i8** }
declare void @exit(i32)
declare i64 @spawn_actor(i8* (i8*)*, i8*)
declare void @actor_send_msg(i64, i64, i8*, i64)
declare %struct.actor_message_struct* @actor_receive(...)
@g1 = constant [4 x i8] c"%d
\00"
define i32 @main(i32 %argc, i8** %argv) {
  %1 = alloca i32
  %2 = alloca i32
  %3 = alloca i8**
  store i32 0, i32* %1
  store i32 %argc, i32* %2
  store i8** %argv, i8*** %3
  call void (...)* @actor_init()
  %4 = call i64 @spawn_actor(i8* (i8*)* bitcast (i8* ()* @_actor_main to i8* (i8*)*), i8* null)
  call void (...)* @actor_wait_finish()
  call void (...)* @actor_destroy_all()
  call void @exit(i32 0)
  unreachable
  %6 = load i32* %1
  ret i32 %6
}
define i8* @_actor_main() {
%_active = alloca i1
store i1 true, i1* %_active
%_msg = alloca %struct.actor_message_struct*
%res = alloca i64
%_1 = add i64 2, 3
%_2 = mul i64 %_1, 4
%_3 = sdiv i64 %_2, 2
store i64 %_3, i64* %res
%_5 = load i64* %res
%_4 = getelementptr [4 x i8]* @g1, i64 0, i64 0
call i32 (i8*, ...)* @printf(i8* %_4, i64 %_5)
store i1 false, i1* %_active
br label %start
br label %start
start:
br label %switch_6
switch_6: 
%_6 = load i1* %_active
br i1 %_6, label %body_6, label %cont_6
body_6: 
%_7 = call %struct.actor_message_struct* (...)* @actor_receive()
store %struct.actor_message_struct* %_7, %struct.actor_message_struct** %_msg
%_8 = load %struct.actor_message_struct** %_msg
%_9 = getelementptr %struct.actor_message_struct* %_8, i32 0, i32 3
%_10 = load i64* %_9
switch i64 %_10, label %start [  ]
br label %switch_6
cont_6:
ret i8* null
ret i8* null
}
\end{lstlisting}

After running a LLVM IR optimiser on the above LLVM IR, the result is as following.

\begin{lstlisting}[breaklines]
@g1 = constant [4 x i8] c"%d\0A\00"
; Function Attrs: nounwind
declare i32 @printf(i8* nocapture readonly, ...) #0
declare void @actor_init(...)
declare void @actor_wait_finish(...)
declare void @actor_destroy_all(...)
declare void @exit(i32)
declare i64 @spawn_actor(i8* (i8*)*, i8*)
; Function Attrs: noreturn
define i32 @main(i32 %argc, i8** nocapture readnone %argv) #1 {
  tail call void (...)* @actor_init()
  %1 = tail call i64 @spawn_actor(i8* (i8*)* bitcast (i8* ()* @_actor_main to i8* (i8*)*), i8* null)
  tail call void (...)* @actor_wait_finish()
  tail call void (...)* @actor_destroy_all()
  tail call void @exit(i32 0)
  unreachable
}
define noalias i8* @_actor_main() {
cont_6.critedge:
  %0 = tail call i32 (i8*, ...)* @printf(i8* getelementptr inbounds ([4 x i8]* @g1, i64 0, i64 0), i64 10)
  ret i8* null
}
attributes #0 = { nounwind }
attributes #1 = { noreturn }
\end{lstlisting}

The optimiser (run with \enquote{opt -O3 -S}) applies constant folding on $(2 + 3 ) * 4 / 2$ such that this expression is evaluated on compile time and $10$ is stored. It also removed all unnecessary operations. While this seems simple, no extra work had to be done on our side, and significant optimizations were achieved.

Running clang (run with \enquote{clang -O3 -lactor -S}) on this optimised LLVM IR, results in the following X86-64 assembly.

\begin{lstlisting}[breaklines]
	.section	__TEXT,__text,regular,pure_instructions
	.macosx_version_min 10, 10
	.globl	_main
	.align	4, 0x90
_main:                                  ## @main
	.cfi_startproc
## BB#0:
	pushq	%rbp
Ltmp0:
	.cfi_def_cfa_offset 16
Ltmp1:
	.cfi_offset %rbp, -16
	movq	%rsp, %rbp
Ltmp2:
	.cfi_def_cfa_register %rbp
	xorl	%eax, %eax
	callq	_actor_init
	leaq	__actor_main(%rip), %rdi
	xorl	%esi, %esi
	callq	_spawn_actor
	xorl	%eax, %eax
	callq	_actor_wait_finish
	xorl	%eax, %eax
	callq	_actor_destroy_all
	xorl	%edi, %edi
	callq	_exit
	.cfi_endproc

	.globl	__actor_main
	.align	4, 0x90
__actor_main:                           ## @_actor_main
## BB#0:                                ## %switch_6
	pushq	%rbp
	movq	%rsp, %rbp
	leaq	_g1(%rip), %rdi
	movl	$10, %esi
	xorl	%eax, %eax
	callq	_printf
	xorl	%eax, %eax
	popq	%rbp
	retq

	.section	__TEXT,__const
	.globl	_g1                     ## @g1
_g1:
	.asciz	"%d\n"

.subsections_via_symbols
\end{lstlisting}