\chapter{Discussion}

This chapter discusses TLDR as a language, the features not implemented in the compiler, and how the language can be further improved in the future.

\section{Language}
The Language Described in this report is a domain specific programming language, borrowing ideas and constructs from other languages. This section will discuss some of these ideas and constructs, and describe TLDRs rationale and usefulness in the domain of programming languages.

\subsection{Distinguishing Features}
One of the main features, and the pièce de résistance of TLDR, is the central role of actors. Actors allow the programmer to model concurrent entities using a fairly simple construct. As opposed to other languages which incorporate actors, actors in TLDR is an independent construct, where in Erlang an actor is a function and Scala (Akka) where actors are just objects. Having actors as a unique construct in TLDR gives the programmer a clue as to the power which actors bring to the language. 

Another difference between TLDR and other languages with actors is that TLDR enforces the use of actors. Every program has at least one actor, main, where the execution of the program starts. This enforces the principle of a central controller in a network of concurrent processes. This means that any TLDR program can be considered, and reasoned about, as a graph-like structure rooted in the main actor, with vertices representing actors and edges representing messages between them.

\subsection{Paradigms}
TLDR focuses a lot on the \enquote{swarm of bees}-style of programming, that is, having large amounts of simple logic rather than having a unified complex algorithm for all the logic in the program. This can be thought of as a characteristic of modular programming languages.

Another focus of TLDR is that of functions as first class citizens. This means that the language does not distinguish between functions and other symbols declared by the programmer. Any place a symbol can be used in a program, a function can be used as well, e.g. as a parameter for a function, a return value of a function or a message to an actor. This, combined with the use of expressions and the avoidance of state and mutable data gives the language a very functional, or declarative, feel, similar to that of F\# or Scala.

The Actor Model does however give the language a modelling form which resembles an object oriented approach. And as will be explained in \cref{dis:func}, creative use of actors can make functions obsolete, which would seem strange for a functional language.

Considering these things, TLDR might not belong strictly to a specific paradigm, but can be adaptive based on programmers approach.

\subsection{Abstraction}
TLDR is a rather high level language. TLDR provides a high level of abstraction over data, with arbitrary precision integer and floating point data types, control flow, with mathematical operations built into the language as well as message passing and loop-structures, and concurrency, with actors and message passing.

\subsection{Representing fractions}\label{dis:frac}

When writing fractions on a single line of text, a common way is to express them as divisions: 1/3, 3/5, 3/10 etc. With this type of notation 1/3 + 2/3 would equal 1. In TLDR, however, the usage of euclidean division would cause the calculation to equal 0, since the two division are evaluated first and both are 0. If TLDR is to consider the divisions as fractions and get the result 1, the fractions must explicitly be noted as real numbers. This could be done with 1.0/3.0 + 2.0/3.0 which would equal 1.0. The reason for this is the principle of not having implicit casting, and also since it would be impossible for a compiler to figure out if euclidean division or fractions is the desired interpretation of the division operator. This limits the languages ability to fulfill the criteria of having traditional mathematical representations of numbers, but it was considered the best alternative.

\section{Compiler}

This section will discuss some language constructs not implemented in the compiler. The full list can be seen in \cref{sec:compilerStatus}.

\subsection{Functions}\label{dis:func}

As the group members were writing programs in TLDR, the absence of functions in the implementation of the compiler, caused some creative ways of modelling a program without the use of functions. As actors were implemented in the compiler, these could be used as a stand-in for functions. A simple function taking a int and returning the int squared, can be modelled as a actor with a receive-method, taking a struct containing an int and a sender actor. 

\begin{lstlisting}[style=TLDR, caption = {Simple example using a function.}]
let f(x) : int -> int := {
  x + x;
};

actor main := {
  receive arguments:args := {
    let res:int := f(2);
    printint(res);
  };
};
\end{lstlisting}
\begin{lstlisting}[style=TLDR, caption = {Example of modeling a function taking a int and returning the int squared, as an actor.}]
actor main := {
  receive arguments:args := {
    let msg:st := {n := 2; sender := me;};
    let handle:a := spawn a msg;
  };
  
  receive res:int := {
    printint(res);
  };
};

actor a := {
  receive msg:st := {
    send msg.sender (msg.n + msg.n);
  };
};

struct st := {
  n:int;
  sender:main;
};
\end{lstlisting}

Theoretically, any function can be modelled as actors taking structures, however there is some mental overhead involved. The function example is shorter to write and arguably also easier to understand.

On the other side, the fact that every function can be modelled as an actor, shows how powerful actors can be in modeling a program into small logical sections.

\subsection{Lists}

The lists implemented in the compiler cannot be reallocated at runtime, and there is no way of concatenating lists. This limits the usefulness of lists, as observed when writing programs in TLDR. Having concatenation of lists implemented in the compiler would allow for several other operations to be implemented based on concatenation. For example, appending an element to a list, can simply be implemented as a concatenation of the list to append to and a list containing 1 element: the element to append. Assuming the concatenation function is implemented, the append function can be implemented as follows.

\begin{lstlisting}[caption = {Implementation of the append function based on the concatenation function.}]
let append(xs, x) : [int] -> int -> [int] := {
  concat(xs, [x]);
};
\end{lstlisting}

\section{Future Language Improvements}

TLDR, being a language focused on allowing easy modeling of real world system, shares a lot of principles with object oriented programming languages, and so the inclusion of common object oriented features is a likely idea. Inheritance would be a good way to allow for faster and more precise programming. Other alternatives could be traits, interfaces or abstract actors. Interfaces could be very useful for making sure that messages are only sent to actors which actually implements a receive for that message, while still allowing multiple types of actors to be candidates. The usefulness of inheritance, or other ways of sharing functionalities across different actors, would allow for more concise code and faster writing.

Another useful language structure which could be implemented in future editions, could be a reply feature. This feature would make it possible to send a message, while evaluating a receive, without explicitly writing which actor will receive said message. Instead the sender of the message being evaluated, will be the target. Besides the possibility for more concise code with such a construct, it would also allow actors to send messages, without having a local reference to their target. This latter point would make it possible for actors to send messages to targets of which they do not know the type of. This makes actors further independent, but at the cost of a more complex compile time check for illegal messages.

\subsection{Support for Discrete Simulations}

Another matter is the support for discrete simulations, as described in \cref{iterprob}. The solutions suggested are based around grouping actors into environments, which work isolated from the rest of the actors. This idea of grouping actors, could potentially bring advantages to the language, such as highly modular programs and functionalities, and promoting purity when writing code in TLDR. But what is more interesting, is the possibility of a change of focus, from having the programmer think in terms of how actors mimic real things, to focusing on having environments mimic the conditions reality sets for things. Examples could be chemistry or physics, where one could want to find the optimal conditions for a chain reaction to occur. This problem could be approached by creating a single version of the actor, and then have the environments implement the changes in values, temperatures, pressure or what it might be. Such an approach might be useful for problems where certain objects behavior can be described through known mathematical structures, but where the conditions can change immensely and by extension, the results of a given simulation.
