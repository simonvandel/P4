\chapter{Introduction}\label{part:introduction}

This report covers the creation of The Language Described in this Report (TLDR), a programming language useful for creating simulations of real world systems.. The language is constructed with scientists, who wish to compute large simulations in different fields of science, as the primary user group. Large simulations refer to problems which are deemed infeasible to be run on regular computers such as laptops and desktops. Focusing on such large simulations effectively gives the language a subfocus on concurrent computing.

In the field of supercomputing, languages such as C and Fortran are heavily used. Although these languages  are very powerful and efficient programming languages, they are beginning to show their age and both lack a lot of prominent functionalities, such as pattern matching, for-each loops and arbitrary precision numerals. 

Instead TLDR tries to provide the computing power of a supercomputer to scientists, who do not have the extended experience with computer science that is required to take full advantage of concurrency and parallelism.

The Language Described in this Report, aims to modernise programming supercomputers and cluster computers.

This premise gives the following problem statement:
\\

\emph{How can one design and implement a programming language which supports a programmer in modeling arbitrary scientific simulations, in a way that inherently encourages concurrent processing.}
