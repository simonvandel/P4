\chapter{Conclusion}


This chapter of the report will conclude on the problem statement presented in \cref{problemSatement}. Based on the problem statement an analysis of the problem domain was made in \cref{problemDomain}. Furthermore, research of concurrency and an analysis of the solution domain resulted in a basis for the design of the functionality of the language in form of criterias for the language described in \cref{analsum}.

The following entail an examination of the way TLDR tries to accomodate the criteria set for the language, as well as the constructed compilers ability to implement the language.

The first criteria was for the language to be simple. This was achieved through staying true to a few select principles, and not allowing them to be broken. The most prominent one being the principle of only letting actors be manipulated through messages. The language also incorporates simplicity in the way that it allows for modeling concurrent programs. The way which the actor model abstracts over the handling of threads in the program removes a lot of technical considerations and thereby simplifies the way the programmer approaches the modeling.

On the subject of orthogonality, the language performs as expected. The focus on simplicity interferes with the orthogonality of the language, but not quite as expected, where the way in which other criteria have interfered. One such criteria could be the lack of implicit casts between number primitives, which was done to uphold readability and the strongly typed language criterias. These criteria have also influenced the language through the explicit difference between the notation for integers and real numbers.

The focus in the syntax design on modeling has also been satisfied. This can be seen in the keywords used for actors.

On the subject of data types and mathematical notation the language strides towards it, but as explained in \cref{dis:frac} sometimes other criterias were prioritized.

Based upon the problem statement, the purpose of this project was to describe how a language for simulation could be designed for concurrent computing. This question has been answered in \cref{part:design}, by specifying the syntax and semantics for the language.
To ensure concurrency, the actor model is the recommended way to do concurrency modeling in TLDR, which helps the programmer to abstract over the specifics of concurrency and parallelism.
The problem statement also asks how the language can be implemented, to answer this, a delimited set of the language has been implemented. \cref{sec:compilerStatus} describes which features were implemented and which ones were not. The section also discusses why some features were prioritized higher than others.

To conclude on the project in general, the compiler and the language documentation answers the problem statement by giving an example of how such a language could be implemented. Although the compiler lacks some features, due to priorities, the compiler is in a working state, where it can compile some code containing fundamental features such as actors. This is assessed to satisfy the criteria for the language.



%simplicitet:
%- fastholdelse af design principper, som er simple
%-- man kan kun sende beskeder til actors, ikke kalde actors
%- simplicitet ved at fjerne muligheder i sproget, som ellers ville øge kompleksiteten
%-- det at skrive parallelle programmer er gjort nemmere ved brug af actors

%ortogonalittet:
%- falder ned når vi ikke bare kan tilfå felterne i en actor, men dette er ok, da det gør sproget simplere
%- funktioner kan sidestilles med variabler. Man kan sende funktioner rundt ligesom variabler/symboler. Dog ikke implementeret

%syntax design:
%- mere målrettet mod modelleringen, og mindre de underliggende tekniske egenskaber
%- spawn keyword: i stedet for at være tekniske, som i alloker, bruges spawn som beskriver hvad der sker mere abstrakt

%data types:
%- vi ville gerne have at matematisk intuition kan bruges i sprog. Det betyder at 2/3 + 1/3 burde give 1, men det vi kun approximerer de irrationelle tal, giver det ikke 1. Dette er en begrænsing vi ikke har ordnet i sproget.
%- andet eksempel med root og potens

%typechecking:
%- vi er strongly typed med klare regler for typer

%readability vs writability:
%- hvor vi har overholdt princippet: man skal skrive en real literal hvis vi gemmer den i en real variabel

%design overfor vores analyse
%- opfylder vores design de krav der er sat i analysen

%compiler overfor vores design
%- opfylder vores compiler de krav der er sat i designet
