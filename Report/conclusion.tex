\chapter{Konklusion}

Hvordan blev kravene til sproget opfyldt?
Hvordan føles det at skrive i sproget?

simplicitet:
- fastholdelse af design principper, som er simple
-- man kan kun sende beskeder til actors, ikke kalde actors
- simplicitet ved at fjerne muligheder i sproget, som ellers ville øge kompleksiteten
-- det at skrive parallelle programmer er gjort nemmere ved brug af actors

ortogonalittet:
- falder ned når vi ikke bare kan tilfå felterne i en actor, men dette er ok, da det gør sproget simplere
- funktioner kan sidestilles med variabler. Man kan sende funktioner rundt ligesom variabler/symboler. Dog ikke implementeret

syntax design:
- mere målrettet mod modelleringen, og mindre de underliggende tekniske egenskaber
- spawn keyword: i stedet for at være tekniske, som i alloker, bruges spawn som beskriver hvad der sker mere abstrakt

data types:
- vi ville gerne have at matematisk intuition kan bruges i sprog. Det betyder at 2/3 + 1/3 burde give 1, men det vi kun approximerer de irrationelle tal, giver det ikke 1. Dette er en begrænsing vi ikke har ordnet i sproget.
- andet eksempel med root og potens

typechecking:
- vi er strongly typed med klare regler for typer

readability vs writability:
- hvor vi har overholdt princippet: man skal skrive en real literal hvis vi gemmer den i en real variabel

design overfor vores analyse
- opfylder vores design de krav der er sat i analysen

compiler overfor vores design
- opfylder vores compiler de krav der er sat i designet